%\documentclass[a4paper,10pt]{article}
\documentclass[english,10pt]{article}

\usepackage[utf8]{inputenc}
\usepackage[russian,english]{babel}
%\usepackage{psfig,amsfonts}
\usepackage{array,longtable}
\usepackage{amssymb}
\usepackage{graphicx} 
\usepackage{graphics}
\setcounter{LTchunksize}{20}
\textheight 23.0cm
\textwidth 18.0cm

\hoffset=-2.5cm
%\voffset=-2.5cm
\baselineskip=22pt
\sloppy

\renewcommand{\refname}{ }
\renewcommand{\tablename}{Таблица}
\renewcommand{\figurename}{Рис.}

\def\bc{\begin{center}}
\def\ec{\end{center}}

\def\lb{\linebreak}
\def\be{\begin{equation}}
\def\ee{\end{equation}}
\def\beq{\begin{eqnarray}}
\def\eeq{\end{eqnarray}}
\def\bfig{\begin{figure}}
\def\efig{\end{figure}}
\def\bnum{\begin{enumerate}}
\def\enum{\end{enumerate}}
\def\ds{\displaystyle}
\def\mm{\mathrm}

\begin{document}

\noindent
Noordermeer E., van der Hulst J. M., 
``The stellar mass distribution in early-type disc galaxies: surface
photometry and bulge-disc decompositions'', 
MNRAS, 376, 1480-1512 (2007)

%*******************************************************DISK********************
\begin{longtable}[c]{cccccccccccccc}
\caption{NGC 3898. Structural parameters of the galaxy} \\ 
\hline 
band & scale & 
$\mu_\mm{e,b}$ & $\mu_\mm{e,b}^\mm{c}$ & $r_\mm{e,b}$ & 
$n$ & $m_\mm{b}$ & $M_\mm{b}$ & 
$\mu_\mm{0,d}$ & $\mu_\mm{0,d}^\mm{c}$ & $h$ & 
$m_\mm{d}$ & $M_\mm{d}$ & $B/D$ \\ 
& (kpc/$''$) & \multicolumn{2}{c}{(mag/$\Box^{''}$)} & ($''$) 
& & \multicolumn{2}{c}{(mag)} & 
\multicolumn{2}{c}{(mag/$\Box^{''}$)} & $''$ & 
\multicolumn{2}{c}{(mag)} & \\
(1)&(2)&(3)&(4)&(5)&(6)&(7)&(8)&(9)&(10)&(11)&(12)&(13)&(14) \\ 
\hline
\endfirsthead 
\hline
\multicolumn{14}{c}{\small\slshape Structural parameters of the galaxy. 
Continued. } \\ \hline
band & scale & 
$\mu_\mm{e,b}$ & $\mu_\mm{e,b}^\mm{c}$ & $r_\mm{e,b}$ & 
$n$ & $m_\mm{b}$ & $M_\mm{b}$ & 
$\mu_\mm{0,d}$ & $\mu_\mm{0,d}^\mm{c}$ & $h$ & 
$m_\mm{d}$ & $M_\mm{d}$ & $B/D$ \\ 
& (kpc/$''$) & \multicolumn{2}{c}{(mag/$\Box^{''}$)} & ($''$) 
& & \multicolumn{2}{c}{(mag)} & 
\multicolumn{2}{c}{(mag/$\Box^{''}$)} & & 
\multicolumn{2}{c}{(mag)} & \\
(1)&(2)&(3)&(4)&(5)&(6)&(7)&(8)&(9)&(10)&(11)&(12)&(13)&(14) \\
\hline
\endhead 
\hline
$R$ & 0.092 & 18.43 & 18.37 & 8.8 & 2.3 & 11.13 & -20.31 & 
19.76 & 20.49 & 36.2 & 10.70 & -20.74 & 0.68 \tabularnewline

$B$ & 0.092 & 19.89 & 19.80 & 8.8 & 2.3 & 12.59 & -18.88 & 
21.30 & 22.00 & 42.9 & 11.92 & -19.55 & 0.54 \tabularnewline
\hline
\end{longtable}

Columns: 
(1) Photometric band. 
(2) Conversion factor to convert arcsecs into kpc.
(3) Bulge effective surface brightness. 
(4) Idem, but corrected for galactic foreground extinction.
(5) Effective radius of the bulge, given in arcsec.
(6) S\`ersic index. 
(7) Bulge total apparent magnitude.
(8) Bulge total absolute magnitude.
(9) Disc central surface brightness. 
(10) Idem, but corrected for galactic foreground extinction.
(11) Disc scalelength, given in arcsec.
(12) Disc total apparent magnitude.
(13) Disc total absolute magnitude. 
(14) The ratio of the bulge to disc luminosities. 

\bigskip
\noindent
Noordermeer E., van der Hulst J.M., Sancisi R., 
Swaters R. S., and van Albada T.S., 
``The mass distribution in early-type disc galaxies: declining rotation
curves and correlations with optical properties'', 
MNRAS, 376, 1513-1546 (2007)

%*******************************************************DISK********************
\begin{longtable}[c]{cccccccccc}
\caption{NGC 3898. Basic data} \\ 
\hline 
Type & D & $M_B$ & $M_R$ & $\mu_\mm{0,d}^\mm{c}$ & $h$ & $r_\mm{e,b}$
& $V_\mm{sys}$ & $PA$ & $i$ \\ 
& (Mpc) & (mag) & (mag) & (mag/$\Box^{''}$) & (kpc) & (kpc)
& (km/s) & (deg) & (deg) \\
(1)&(2)&(3)&(4)&(5)&(6)&(7)&(8)&(9)&(10) \\ 
\hline
\endfirsthead 
\hline
\multicolumn{10}{c}{\small\slshape Basic data. 
Continued. } \\ \hline
Type & D & $M_B$ & $M_R$ & $\mu_\mm{0,d}^\mm{c}$ & $h$ & $r_\mm{e,b}$
& $V_\mm{sys}$ & $PA$ & $i$ \\ 
& (Mpc) & (mag) & (mag) & (mag/$\Box^{''}$) & (kpc) & (kpc)
& (km/s) & (deg) & (deg) \\
(1)&(2)&(3)&(4)&(5)&(6)&(7)&(8)&(9)&(10) \\ 
\hline
\endhead 
\hline
SA(s)ab & 18.9 & -20.00 & -21.28 & 20.49 & 3.3 & 0.8 
& 1172 & 107-118 & 69-66 \tabularnewline
\hline
\end{longtable}

Columns: 
(1) Morphological type from NED). 
(2) Distance. 
(3), (4) absolute B-and R-band magnitudes 
(corrected for Galactic foreground extinction).
(5) R-band central disc surface brightness 
(corrected for Galactic foreground extinction and inclination effects).
(6) $R$-band disc scalelength.
(7) $R$-band bulge effective radius.
(8) Heliocentric systemic velocity.
(9) Position angle (north through east) of major axis. 
(10)Inclination angle.

\bigskip
\noindent
M\'{e}ndez-Abreu J., Aguerri J. A. L., Corsini E. M., 
and Simonneau E., 
``Structural properties of disk galaxies. I. The intrinsic 
equatorial ellipticity of bulges'', 
A\&A, 478, 353-369 (2008)

%*******************************************************DISK********************
\begin{longtable}[c]{cccccccccccc}
\caption{NGC 3898. Structural parameters of the galaxy} \\ 
\hline 
band & D & $V_{3K}$ & 
$\mu_\mm{e,b}$ & $r_\mm{e,b}$ & $n$ & $q_\mm{b}$ & PA$_\mm{b}$ & 
$\mu_\mm{0,d}$ & $h$ & $q_\mm{d}$ & PA$_\mm{d}$ \\ 
& (Mpc) & (km/s) & 
(mag/$\Box^{''}$) & ($''$) & & & (deg) & 
(mag/$\Box^{''}$) & ($''$) & & (deg) \\
(1)&(2)&(3)&(4)&(5)&(6)&(7)&(8)&(9)&(10)&(11)&(12) \\ 
\hline
\endfirsthead 
\hline
\multicolumn{12}{c}{\small\slshape Structural parameters of the galaxy. 
Continued. } \\ \hline
band & D & $V_{3K}$ & 
$\mu_\mm{e,b}$ & $r_\mm{e,b}$ & $n$ & $q_\mm{b}$ & PA$_\mm{b}$ & 
$\mu_\mm{0,d}$ & $h$ & $q_\mm{d}$ & PA$_\mm{d}$ \\ 
& (Mpc) & (km/s) & 
(mag/$\Box^{''}$) & ($''$) & & & (deg) & 
(mag/$\Box^{''}$) & ($''$) & & (deg) \\
(1)&(2)&(3)&(4)&(5)&(6)&(7)&(8)&(9)&(10)&(11)&(12) \\
\hline
\endhead 
\hline
$J$ & 21.9 & 1340 &
18.13 & 11.9 & 3.75 & 0.64 & 107.9 & 
19.07 & 29.2 & 0.5 & 106.9 \tabularnewline
\hline
\end{longtable}

Columns:
(1) Photometric band.
(2) Distance, obtained as $V_{3K}/H_0$ with $H_0=$75 km\,s$^{-1}$.
(3) Radial velocity with respect to the CMB from LEDA.
(4) Bulge effective surface brightness. 
(5) Effective radius of the bulge, given in arcsec.
(6) S\`ersic index. 
(7) Axis ratio of the bulge.
(8) Position angle of the bulgescale lenth of the disc.
(9) Disc central surface brightness. 
(10) Disc scalelength, given in arcsec.
(11) Axis ratio of the disc.
(12) Position angle of the disc. 

\newpage
\noindent
Guti\'{e}rrez Leonel, Erwin Peter, Aladro Rebeca, and 
Beckman John E., 
``THE OUTER DISKS OF EARLY-TYPE GALAXIES. 
II. SURFACE-BRIGHTNESS PROFILES OF UNBARRED GALAXIES AND TRENDS 
WITH HUBBLE TYPE'', 
ApJ, 142, 145(31pp) (2011)

%*******************************************************DISK********************
\begin{longtable}[c]{ccccccccccc}
\caption{NGC 3898. Structural parameters of the galaxy} \\ 
\hline 
band & scale & $D$   & PA    & $i$ &
$h_\mm{d,inner}$ & $h_\mm{d,outer}$ & $R_\mm{d,break}$ &
$\mu_\mm{0,d,inner}$ & $\mu_\mm{0,d,outer}$ & $\mu_\mm{0,d,break}$ \\ 
     &       & (Mpc) & (deg) & (deg) &
($''$)           & ($''$)           & ($''$)           & 
(mag/$\Box^{''}$)    & (mag/$\Box^{''}$)    & (mag/$\Box^{''}$) \\
(1)&(2)&(3)&(4)&(5)&(6)&(7)&(8)&(9)&(10)&(11) \\ 
\hline
\endfirsthead 
\hline
\multicolumn{11}{c}{\small\slshape Structural parameters of the galaxy. 
Continued. } \\ \hline
band & scale & $D$   & PA    & $i$ &
$h_\mm{d,inner}$ & $h_\mm{d,outer}$ & $R_\mm{d,break}$ &
$\mu_\mm{0,d,inner}$ & $\mu_\mm{0,d,outer}$ & $\mu_\mm{0,d,break}$ \\ 
     &       & (Mpc) & (deg) & (deg) &
($''$)           & ($''$)           & ($''$)           & 
(mag/$\Box^{''}$)    & (mag/$\Box^{''}$)    & (mag/$\Box^{''}$) \\
(1)&(2)&(3)&(4)&(5)&(6)&(7)&(8)&(9)&(10)&(11) \\
\hline
\endhead 
\hline
$J$ & 0.092 & 18.3 & 107 & 53 & 
30.0 & 59.9 & 111 & 
19.54 & 21.53 & 23.3 \tabularnewline
\hline
\end{longtable}

Columns: 
(1) Photometric band. 
(2) Conversion factor to convert arcsecs into kpc.
(3) Distance.
(4), (5) Position angle and inclination of the outer disc.
(6), (7) Scale length for the inner and outer exponential fits, 
respectively. 
(8) Position of break point on the profile. 
(9), (10) Central $R$-band surface brightness 
for the inner and outer exponential fits, respectively. 
(11) Surface brightness at the break point.

\bigskip
\noindent
Фотометрия в $B$ и $R$ даёт согласованные значения центральной 
поверхностной плотности для диска, но они не очень большие 
при отношении $(M/L)_R \approx 2$ (по цвету $B-R$). Есть динамическая 
оценка $(M/L)_R$ (из кривой вращения, Noordermeer, thesis) --- 
от 1.5 до 4 (4 --- для ``максимального'' диска), если используется 
NFW модель тёмного гало.

\bigskip
\noindent
Тем не менее фотометрия в полосе $J$ (M\'{e}ndez-Abreu et al., 2008) 
согласуется с фотометрией в оптике, т.е. приводит к тем же значениям 
поверхностной плотности диска, что и для полосы $R$ 
(порядка 330 $M_\odot/$пк$^2$).

\bigskip
\noindent
Декомпозиция галактики немного противоречивая, но, по-видимому, 
у неё есть слабый протяженный внешний диск с большой шкалой.

\bigskip
\noindent
Галактика довольно хорошо изучена. Есть излучение в H$_\alpha$, 
но SFR низкое.

\bigskip
\noindent
В области вне яркого балджа видны туго закрученные спирали, в них 
голубые уярчения (SDSS). 

\bigskip
\noindent
На профиле распределения плотности водорода выделяются два кольца 
($R \simeq 70''$ и $R \simeq 200''$).

\end{document}
