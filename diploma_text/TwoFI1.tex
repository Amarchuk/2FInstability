%\documentclass[a4paper,12pt]{article}
\documentclass[russian,12pt]{article}

\usepackage[utf8]{inputenc}
\usepackage[russian]{babel}
\usepackage[]{natbib}
%\usepackage{psfig,amsfonts}

\textheight 22.0cm
\textwidth 16.0cm

\hoffset=-1.0cm
\voffset=-0.2cm
\baselineskip=24pt
\sloppy

\def\bs{\bigskip}
\def\noi{\noindent}

\def\bc{\begin{center}}
\def\ec{\end{center}}
\def\be{\begin{equation}}
\def\ee{\end{equation}}

\begin{document}

\bc
\Large\bf 
Двухжидкостная неустойчивость и крупномасштабное звёздообразование \\
\vspace{0.5cm}
{\it Н.Я.~Сотникова}
\ec

\section{Критерий гравитационной неустойчивости}

Точная оценка минимальной дисперсии скоростей в радиальном направлении, 
необходимой для локальной устойчивости звёздного диска относительно 
осесимметричных возмущений, получается из анализа бесстолкновительного 
уравнения Больцмана 
\citep{Toomre64}.

\noi
Для бесконечно тонкого звёздного диска
\be
\sigma_{R}^\mathrm{min} = \frac{3.36 \, G \, \Sigma_\mathrm{d}}{\kappa} \, ,
\label{toomre_disp}
\ee
где $\Sigma_\mathrm{s}$ --- поверхностная плотность звёзд диска, а 
$\displaystyle \kappa = 
\left( \frac{3}{R} \frac{\partial \Phi}{\partial R} + 
\frac{\partial^2 \Phi}{\partial R^2} \right) $ 
--- эпициклическая частота, при этом $\Phi$ 
--- гравитационный потенциал, определяемый всеми подсистемами 
галактики.

Эпициклическую частоту можно выразить через круговую скорость вращения 
$v_\mathrm{c}(R)$
$$
\displaystyle \kappa = 
\sqrt{2 \left(\frac{v_\mathrm{c}^2}{R^2} + 
\frac{v_\mathrm{c}}{R} \frac{d v_\mathrm{c}}{dR} \right)} \, .
$$
Если известна кривая вращения, то её можно аналитически сгладить и, 
вычислив производную, получить профиль $\kappa(R)$.

Критерий (\ref{toomre_disp}) можно представить по-другому --- через 
критическое значение поверхностной плотности
\be
\Sigma_\mathrm{s}^\mathrm{cr} = \frac{\kappa \, \sigma_R}{3.36 \, G} \, .
\label{toomre_dens}
\ee

\noi
Если поверхностная плотность звёзд диска 
$\Sigma_\mathrm{s} > \Sigma_\mathrm{s}^\mathrm{cr}$, то диск неустойчив 
относительно осесимметричных возмущений. Это один из основных 
результатов работы \cite{Toomre64}.

Для газового диска аналогичный критерий получается из анализа 
гидродинамических уравнений \citep{GL_B65}
\be
\Sigma_\mathrm{g}^\mathrm{cr} = \frac{\kappa \, c_\mathrm{g}}{\pi \, G} \, ,
\label{GL_dens}
\ee
где $c_\mathrm{g}$ --- скорость звука в газе. Фактически выражения 
(\ref{toomre_dens}) и (\ref{GL_dens}) отличаются только коэффициентами: 
3.36 --- в формуле~(\ref{toomre_dens}) и $\pi$ --- в выражении~(\ref{GL_dens}).

Для измерения запаса прочности диска относительно роста возмущений 
плотности используют специальный параметр --- число Тумре $Q\mathrm{s}$. 
Для звёздного диска
\be
Q\mathrm{s} = \frac{\sigma_R}{\sigma_{R}^\mathrm{min}} = 
\frac{\kappa \, \sigma_R}{3.36 \, G \, \Sigma_\mathrm{s}} \, .
\label{toomre_par}
\ee
Для устойчивости относительно осесимметричных возмущений необходимо 
$Q\mathrm{s} >1$. 

\section{Критерий гравитационной неустойчивости и крупномасштабное 
звёздообразование в дисковых галактиках}

Темп звёздообразования в дисковых галактиках связывают с локальной плотностью 
газа в диске.

\subsection{Закон Шмидта}
Часто текущий темп звёздообразования на масштабах всей галактики 
$\psi(t)$ описывается эмпирическим законом Шмидта \citep{Schmidt59}
\be
\psi(t) = \frac{d {\cal M}_\mathrm{s}}{dt} \propto \Sigma_\mathrm{g}^{k} \, ,
\label{schmidt}
\ee
где ${\cal M}_\mathrm{s}$ --- масса звезд, $\Sigma_\mathrm{g}$ ---
поверхностная плотность газа, $k \approx 2$. Качественно этот закон 
обосновывается следующим образом. Звёзды рождаются в газовых облаках, 
коллапсирующих под действием гравитационной (джинсовской) неустойчивости. 
Процессы охлаждения играют ключевую роль в такого рода сжатии. Скорость 
охлаждения во многих случаях пропорциональна квадрату концентрации частиц 
в газе $n_{\rm g}^{2}$ (ударное охлаждение), поэтому можно ожидать появления 
квадратичной (или во всяком случае нелинейной) зависимости между скоростью 
звёздообразования и поверхностной плотностью газа.

\subsection{Физическое обоснование закона Шмидта}
\cite{Kennicutt89} исследовал закон Шмидта для большой выборки галактик. 
Индикатором звёздообразования служила интенсивность излучения в линии 
$\mathrm{H}_{\alpha}$. Для тех областей галактик, в которых газа много, 
он получил
\be
I(\mathrm{H}_{\alpha}) \propto \Sigma_\mathrm{g}^{1.3} \, .
\label{schmidt2}
\ee
На больших расстояниях от центра галактики интенсивность излучения
$I(\mathrm{H}_{\alpha})$ быстро падает ниже величины, предсказываемой
соотношением (\ref{schmidt2}). Радиус области, где найденный
эмпирический закон выполняется, совпадает с радиусом области, где
величина поверхностной плотности газа лежит выше значения
\be
\Sigma_\mathrm{g}^\mathrm{cr89} = \alpha \frac{\kappa c_\mathrm{g}}{\pi G} \, .
\label{kennicutt89}
\ee
Для скорости звука в газе $c_\mathrm{g}$ \cite{Kennicutt89} брал 
значение $\sim 6$ км с$^{-1}$). 

Без множителя $\alpha$ критерий~(\ref{kennicutt89}) фактически совпадает 
с критерием~(\ref{GL_dens}). \cite{Kennicutt89} эмпирически получил 
значение коэффициента $\alpha = 0.67$ и дал физическое объяснение 
этому. Классический критерий Тумре (\ref{toomre_dens}), также как и 
критерий~(\ref{GL_dens}), применимы только 
для осесимметричных возмущений. Так как в диске всегда присутствуют 
возмущения и других мод (неосесимметричные), то диску нужно иметь 
больший запас прочности, чтобы оставаться устойчивым. Таким образом, 
при $\Sigma_\mathrm{g} > \Sigma_\mathrm{g}^\mathrm{cr89}$ диск
гравитационно неустойчив относительно неосесимметричных возмущений. 
И в тех областях, где газовый диск неустойчив, существует нелинейная 
зависимость (\ref{schmidt2}) между скоростью звёздообразования и поверхностной 
плотностью газа. 

Нелинейная пороговая связь между скоростью звёздообразования и 
гравитационной неустойчивостью звёздного диска неоднократно 
исследовалась и подтверждалась во многих работах.

В работе \cite{MK01} изучались области звёздообразования по 
$\mathrm{H}_{\alpha}$ CCD фотометрии в 32 близких галактиках с хорошо 
измеренными кривыми вращения и известными значениями профилей поверхностной 
плотности HI и H$_2$. Практически для всех галактик резкий обрыв на профиле 
излучения в линии $\mathrm{H}_{\alpha}$ наблюдался в тех областях, где 
газовый диск становился устойчивым согласно критерию (\ref{kennicutt89}).

\section{Критерий гравитационной неустойчивости в двухжидкостном 
приближении}

В присутствии звёздного диска газовый диск, будучи маржинально 
устойчивым сам по себе, тем не менее может перейти в неустойчивое состояние. 
В сугубо гидродинамическом приближении задача о неустойчивости двух 
жидкостей была рассмотрена \cite{JS84}. Было выведено дисперсионное 
уравнение для осесимметричных возмущений, а критерий неустойчивости 
сформулирован в следующем виде
\begin{equation}
\frac{2\,\pi\,G\,k\,\Sigma_\mathrm{s}}{\kappa+k^2 \sigma_\mathrm{s}} + 
\frac{2\,\pi\,G\,k\,\Sigma_\mathrm{g}}{\kappa+k^2 c_\mathrm{g}} > 1
\label{2fluidJS1}
\end{equation}
для всех волновых чисел $k$. В выражении (\ref{2fluidJS1}) 
$\sigma_\mathrm{s}$ --- дисперсия скоростей звёзд. Можно ввести 
обозначения: безразмерный параметр Тумре для звёздного 
диска\footnote{Здесь параметр Тумре отличается от значения, даваемого 
выражением~(\ref{toomre_par}) для бесстолкновительных систем. Это 
означает, что при введённых обозначениях однокомпонентный звёздный диск 
неустойчив, когда $Q_\mathrm{s} < 3.36/\pi = 1.07$.}
$\displaystyle Q_\mathrm{s} \equiv 
\frac{\kappa\,\sigma_\mathrm{s}}{\pi\,G\,\Sigma_\mathrm{s}}$, 
безразмерный параметр Тумре для газового диска  
$\displaystyle Q_\mathrm{g} \equiv 
\frac{\kappa\,c_\mathrm{g}}{\pi\,G\,\Sigma_\mathrm{g}}$, безразмерное 
волновое число 
$\displaystyle \bar{k} \equiv \frac{k\,\sigma_\mathrm{s}}{\kappa}$, 
отношение скорости звука в газе к дисперсии скоростей звёзд 
$\displaystyle s \equiv \frac{c_\mathrm{g}}{\sigma_\mathrm{s}}$. 
Преобразуем левую часть выражения~(\ref{2fluidJS1}) с учётом введённых 
обозначений и запишем выражение для эффективного параметра Тумре 
$Q_\mathrm{eff}$
$$
\frac{2}{Q_\mathrm{s}} \frac{\bar{k}}{1 + \bar{k}^2} + 
\frac{2}{Q_\mathrm{g}} s \frac{\bar{k}}{1 + \bar{k}^2 s^2} = 
\frac{1}{Q_\mathrm{eff}} \, .
$$
Тогда условие неустойчивости двухжидкостной сведётся к следующему 
неравенству
$$
\frac{1}{Q_\mathrm{eff}} > 1 \, , 
\,\, \mbox{или} \,\,\,\,\,\, Q_\mathrm{eff} < 1 
$$
для всех значений безразмерного волнового числа $\bar{k}$.
Основной результат работы \cite{JS84} состоит в том, что при 
определённых условиях, даже если $Q_\mathrm{g} > 1$ (газовый диск 
устойчив) и $Q_\mathrm{s} > 1$ (звёздный диск устойчив), 
звёздно-газовый диск может оказаться неустойчивым.

Условие $Q_\mathrm{eff} < 1$ эквивалентно условию 
\begin{equation}
\frac{2}{Q_\mathrm{s}} \frac{\bar{k}}{1 + \bar{k}^2} + 
\frac{2}{Q_\mathrm{g}} s \frac{\bar{k}}{1 + \bar{k}^2 s^2} > 1 \, .
\label{2fluidJS2}
\end{equation}
\cite{Elmegreen95} ясно показал, что это условие со знаком равенства 
сводится к решению кубического уравнения относительно $\bar{k}$, а 
неравенство (\ref{2fluidJS2}) эквивалентно тому, что максимум 
выражения, стоящего слева и являющегося функцией $\bar{k}$, 
оказывается больше~1. Как правило, у этого выражения один максимум 
(один действительный корень для кубического уравнения, получающегося 
путём дифференцирования выражения слева и приравнивания производной 
нулю). Однако, для некоторых значений параметров 
$s$ и $\displaystyle q \equiv \frac{Q_\mathrm{g}}{Q_\mathrm{s}}$ может 
существовать два максимума (два действительных корня соответствующего 
кубического уравнения --- $\bar{k}_\mathrm{m1}$ и 
$\bar{k}_\mathrm{m2}$). Тогда условие (\ref{2fluidJS2}) должно 
выполняться для $\bar{k}_\mathrm{m}$, дающего больший максимум. 
\cite{BR88} численно показали, что у выражения, стоящего в левой части 
(\ref{2fluidJS2}), имеется один максимум, если $s^2 > 0.0294$ и 
$\Sigma_\mathrm{g}/\Sigma_\mathrm{s} > 0.172$. 

\cite{Efstathiou00} записал в удобной форме выражение для критической 
плотности газа в присутствии звёздного диска аналогично выражению 
(\ref{GL_dens})
\begin{equation}
\Sigma_\mathrm{g}^\mathrm{cr,2} = 
\frac{\kappa c_\mathrm{g}}{\pi\, G\, g(a,b)} \, ,
\label{2fluid}
\end{equation}
где $a = \sigma_\mathrm{s}/c_\mathrm{g}$ --- отношение дисперсии скоростей 
звёзд к к скорости звука в газе (или дисперсии скоростей облаков в газе; 
если следовать уже введённым обозначениям, то $a = 1/s$), 
$b = \Sigma_\mathrm{s}/\Sigma_\mathrm{g}$ --- отношение поверхностных 
плотностей звёздного и газового дисков, 
а $g(a,b)$~---~функция, численно рассчитанная в работе \citet{Efstathiou00} 
(её вычисление сводится к решению опять же кубического уравнения). 
Так как значение $g(a,b)$ больше~1 для любых значений $a$ и $b$, то 
неустойчивость в присутствии звёздного диска наступает при меньших значениях 
поверхностной плотности газового диска, чем по критерию 
(\ref{GL_dens}). Это существенно для астрофизических приложений в 
контексте исследования динамического статуса галактических дисков и 
процесса крупномасштабного звёздообразования.

\cite{WS94} предложили простую аппроксимационную формулу для 
эффективного значения параметра Тумре $Q_\mathrm{eff}$ в виде
\begin{equation}
\frac{1}{Q_\mathrm{eff}} = \frac{1}{Q_\mathrm{s}} + \frac{1}{Q_\mathrm{g}} \, .
\label{WS}
\end{equation}
В случае неустойчивости значение $Q_\mathrm{eff}$ должно быть меньше~1.

\cite{Jog96} и \cite{RW11} раскритиковали это выражение. 
\cite{RW11} показали, что использование этой формулы может приводить к 
ошибкам в оценках $Q_\mathrm{eff}$ до 50\%. Они также привели и обосновали 
свою аппроксимационную формулу, дающую погрешность не более 9\%,
\be
\frac{1}{Q_\mathrm{eff}} = 
\left\{
\begin{array}{rc}
\displaystyle \frac{W}{Q_\mathrm{s}} + \frac{1}{Q_\mathrm{g}} \, , 
& Q_\mathrm{s} \geq Q_\mathrm{g}\, , \\
& \\
\displaystyle \frac{1}{Q_\mathrm{s}} + \frac{W}{Q_\mathrm{g}} \, , 
& Q_\mathrm{s} \leq Q_\mathrm{g}\,  ; \\
\end{array}
\right.
\label{RW11}
\ee
где $\displaystyle W = \frac{2s}{1 + s^2}$, а 
$\displaystyle s = \frac{c_\mathrm{g}}{\sigma_\mathrm{s}}$. Если 
известны значения $\sigma_\mathrm{s}$, $c_\mathrm{g}$, $\Sigma_\mathrm{s}$ 
и $\Sigma_\mathrm{g}$, то можно сделать оценку $Q_\mathrm{eff}$ для 
данной области диска и определить, устойчив или неустойчив диск 
локально в этой области.

На самом деле все эти аппроксимации имеют не слишком большое значения. 
Во-первых, современные вычислительные средства легко позволяют численно 
находить и значение $Q_\mathrm{eff}$ в зависимости от заданных параметров 
$\sigma_\mathrm{s}$, $c_\mathrm{g}$, $\Sigma_\mathrm{s}$, 
$\Sigma_\mathrm{g}$, и значение критической поверхностной плотности, 
например, газового диска. Во-вторых, критерий (\ref{2fluidJS1}) был 
сформулирован в гидродинамическом приближении. Для звёздного диска это 
приближение верно только для длинноволновых возмущений. В общем случае для 
описания звёздного диска нужно брать бесстолкновительное уравнение 
Больцмана (как это сделано в работе\citep{Toomre64}), а не гидродинамические 
уравнения. При этом изменяется форма соответствующего дисперсионного 
уравнения. Корректное рассмотрение неустойчивости звёздно-газового диска 
провёл \cite{Rafikov01} и получил следующее выражение для условия 
гравитационной неустойчивости
\begin{equation}
\frac{2}{Q_\mathrm{s}} \frac{1}{\bar{k}} 
\left[1 - e^{-\bar{k}^2} I_0(\bar{k}^2)\right] + 
\frac{2}{Q_\mathrm{g}} s \frac{\bar{k}}{1 + \bar{k}^2 s^2} > 1 \, ,
\end{equation}
где $I_0(\bar{k}^2)$ --- модифицированная функция Бесселя первого рода.
Условие выполнения этого неравенства уже не сводится к решению 
кубического уравнения и должно находиться численно. Для этого нужно 
при заданных параметрах $\sigma_\mathrm{s}$, $c_\mathrm{g}$, 
$\Sigma_\mathrm{s}$ и $\Sigma_\mathrm{g}$ 
(соответственно при заданных значениях $Q_\mathrm{s}$, $Q_\mathrm{g}$ и 
$s = \sigma_\mathrm{g}/c_\mathrm{s}$) найти все максимумы выражения 
в левой части и убедиться, что самое максимальное значение больше~1. 
Это будет означать, что существуют возмущения определённых длин волн, 
относительно которых звёздно-газовый диск неустойчив. В противном случае 
диск устойчив.

\section{Учёт конечной толщины звёздного и газового дисков}

Для реалистичных галактических дисков при применении к ним 
двухжидкостного критерия гравитационной неустойчивости необходимо ещё 
учитывать толщину газового и звёздного дисков. Эффект конечной толщины 
проявляется в том, что уменьшаается плотность вблизи $z = 0$, как 
следствие, уменьшается гравитационный потенциал, и диск приобретает 
дополнительный запас прочности относительно возмущений плотности. В 
своей первой работе \cite{JS84} показали, как следует модифицировать 
дисперсионное уравнение и соответствующий критерий неустойчивости 
(\ref{2fluidJS1}). Если обозначить характерный вертикальный масштаб 
звёздного диска через $2 h_z^\mathrm{s}$, а газового --- через 
$2 h_z^\mathrm{g}$, то условие неустойчивости следует переписать в виде
\begin{equation}
\frac{2\,\pi\,G\,k\,\Sigma_\mathrm{s}}{\kappa+k^2 \sigma_\mathrm{s}} \,
\left\{
\frac{\left[1 - \exp(-k\,h_z^\mathrm{s})\right]}
{k\,h_z^\mathrm{s}}
\right\} 
+ 
\frac{2\,\pi\,G\,k\,\Sigma_\mathrm{g}}{\kappa+k^2 c_\mathrm{g}} \,
\left\{
\frac{\left[1 - \exp(-k\,h_z^\mathrm{g})\right]}
{k\,h_z^\mathrm{g}}
\right\} 
> 1\, ,
\label{2fluidJS1z}
\end{equation}

\cite{Romeo92} подробно проанализировал эффект конечной толщины, а 
\cite{Elmegreen95} в приближении, предложенном \cite{JS84}, рассмотрел 
несколько примеров и показал, что конечная толщина звёздного и газового 
дисков увеличивает значение эффективного параметра Тумре примерно на 
20\%, давая дополнительный запас прочности диску.

\cite{Romeo94} рассмотрел этот вопрос в деталях и предложил простой 
поправочный коэффициент $T$ для оценки параметра устойчивости диска
\be
T \approx 0.8 + 0.7 \left( \frac{\sigma_z}{\sigma_R} \right) \, ,
\ee
где $\sigma_z$ --- дисперсия скоростей в вертикальном направлении 
(связанная очевидным образом с полутолщиной диска), а $\sigma_R$ --- 
дисперсия скоростей в радиальном направлении. Как показано в работе 
\cite{Romeo94}, эта формула применима для значений отношения дисперсий 
в пределах $0.5 \leq \sigma_z / \sigma_R \leq 1$. Обычно это справедливо для 
звёздных дисков (в окрестности Солнца это отношение равно 
$0.53 \pm 0.07$, \citealp{DB98}). 
Очевидно, что для газового диска это отношение следует брать равным~1 
(здесь справедлива изотропность). \cite{RW11} предложили удобную 
аппроксимационную формулу для $Q_\mathrm{eff}$, дающую погрешность не более 
15\% по сравнению с численной оценкой,
\be
\frac{1}{Q_\mathrm{eff}} = 
\left\{
\begin{array}{rc}
\displaystyle \frac{W}{T_\mathrm{s}\,Q_\mathrm{s}} + 
\frac{1}{T_\mathrm{g}\,Q_\mathrm{g}} \, , 
& T_\mathrm{s}\,Q_\mathrm{s} \geq T_\mathrm{g}\,Q_\mathrm{g}\, , \\
& \\
\displaystyle \frac{1}{T_\mathrm{s}\,Q_\mathrm{s}} + 
\frac{W}{T_\mathrm{g}\,Q_\mathrm{g}} \, , 
& T_\mathrm{s}\,Q_\mathrm{s} \leq T_\mathrm{g}\,Q_\mathrm{g}\,  ; \\
\end{array}
\right.
\label{RW11z}
\ee

\section{Астрофизические приложения}

В астрофизических приложениях критерий двухжидкостной неустойчивости 
использовался в разных контекстах. Одной из областей его приложений 
является вопрос о регулировании крупномасштабного звёздообразования. 
\cite{MK01} попытались оценить влияние звёздного диска на условие 
наступления неустойчивости в газовом диске. В приближении \cite{WS94} 
(см. формулу~(\ref{WS})) они сделали оценку $Q_\mathrm{eff}$ для нашей 
Галактики и нашли, что в солнечной окрестности это значение оказывается 
равным примерно $Q_\mathrm{eff} = \alpha_\mathrm{eff} Q_\mathrm{g}$, 
где $\alpha_\mathrm{eff} = 0.72$. Для рассмотренной ими выборки они 
нашли, что $\alpha_\mathrm{eff}$ варьируется в пределах от 0.25 до 1.0. 
Это означает, что в некоторых случаях запас прочности газового диска 
может быть преувеличен в несколько раз, если не учитывается влияние 
звёздного диска!

\cite{Boissier+03} для выборки из 16 галактик с известными кривыми 
вращения вычислили радиальные профили отношения 
$\Sigma_\mathrm{g} / \Sigma_\mathrm{g}^\mathrm{cr}$. 
Величина $\Sigma_\mathrm{g}^\mathrm{cr}$ вычислялась в приближении 
\cite{WS94}
\be
\Sigma_\mathrm{g}^\mathrm{cr,WS} = \gamma 
\frac{\kappa \, c_\mathrm{g}}{\pi \, G} \, ,
\label{B03}
\ee
где
$$
\gamma = \left( 
1 + 
\frac{\Sigma_\mathrm{s}\,c_\mathrm{g}}{\Sigma_\mathrm{g}\,\sigma_\mathrm{s}} 
\right)^{-1} \, .
$$
Во всех случаях, включая отдельно рассмотренный пример с нашей Галактикой, эти 
профили оказались лежащими выше тех, что получаются по критерию 
одножидкостной неустойчивости~(\ref{GL_dens}). Величина отношения 
$\Sigma_\mathrm{g} / \Sigma_\mathrm{g}^\mathrm{cr}$ близка к~1 (или 
больше), примерно постоянна в областях интенсивного звёздообразования и 
резко падает вне этих областей. Результат в целом подтверждает выводы 
работ \cite{Kennicutt89} и \cite{MK01}, но получен более корректным 
образом с учётом дестабилизирующего влияния звёздного диска.

Тем не менее результат работы \cite{Boissier+03} нельзя считать вполне 
удовлетворительным. 

Во-первых, использованное приближение \cite{WS94} плохое. 

Во-вторых, профиль $\sigma_\mathrm{s}(R)$ определялся не из спектральных 
данных, а из эмпирической линейной зависимости между дисперсией скоростей 
звёзд на расстоянии от центра, равном одному экспоненциальному масштабу 
звёздного диска $h_B$ (обычно в полосе $B$), и круговой скоростью вращения 
$v_\mathrm{c}$ \citep{Bottema93}. Зависимость имеет довольно большой 
разброс и вносит неопределённость в оценку 
$\Sigma_\mathrm{g}^\mathrm{cr}$. Форма профиля предполагалась 
экспоненциальной с масштабом, равным двум масштабам звёздного диска 
$h_B$. Второе предположение имеет под собой физическое обоснование, но 
оно не обязательно верно для конкретных галактик. Более того, в этой 
работе совершенно не учитывался эффект конечной толщины дисков.

Очень интересный подход к исследованию связи между крупномасштабным 
звёздообразованием и динамическим состоянием газового диска 
продемонстрирован в работе \cite{Yang+07}. Изучалось Большое 
Магелланово Облако (БМО). Объект был выбран в силу близости к нашей Галактике 
и возможности проводить звёздную (а не поверхностную) фотометрирю с 
высоким разрешением. Внутреннее и внешнее поглощение для БМО малы, а 
умеренный угол наклона галактики к лучу зрения позволяет практически 
однозначно интерпретировать кинематические данные. 

Поверхностная плотность звёзд определялась по плотности красных гигантов 
и звёзд асимптотической ветви, которые хорошо трассируют старое звёздное 
население, фактически полностью определяющее вклад всех звёзд в 
гравитационный потенциал галактики. Анализ их кинематики давал оценку 
дисперсии скоростей. Данные с космического телескопа 
Спитцер использовались для картографирования молодых звёздных объектов, 
которые являются индикаторами текущего звёздообразования. Также 
использовались карты распределения нейтрального и молекулярного 
водорода. 

Все эти данные в совокупности использовались для картографирования величины 
$Q_\mathrm{eff}$, вычисляемой по формуле \cite{Rafikov01} 
(т.е. дестабилизирующее влияние звёздного диска учитывалось корректно, 
хотя эффект конечной толщины диска не принимался во внимание). 
На карту распределения молодых объектов были нанесены линии одинаковых 
значений значений $Q_\mathrm{eff}$. Области сгущений молодых объектов 
всегда лежали внутри контуров $Q_\mathrm{eff} < 1.0$, а их количество 
резко падало вне этих контуров. Для сравнения вычислялась величина 
$Q_\mathrm{g}$ (без учёта влияния звёздного диска). Сделан вывод, что 
пренебрегать этим влиянием нельзя.

Самое обширное исследование связи крупномасштабного звёздообразования с 
различными процессами в газовом диске было проведено \cite{Leroy+08}. 
В этой работе упор делался на такие процессы, как гравитационная 
неустойчивость, разрушение гигантских молекулярных 
облаков (ГМО) дифференциальным вращением (galactic shear), а также 
тепловая неустойчивость и молекуляризация образующихся в результате 
этого холодных облаков. Все эти процессы относятся к пороговм.  
Они включают механизм звёздообразования при достижении газом 
определённых значений физических параметров. Рассматривалась выборка 23 
близких спиральных и неправильных галактик. 
Использовались очень хорошие наблюдательные данные. 

Карты HI были взяты из обзора THINGS 
(The HI Nearby Galaxy Survey; \citealp{Walter+08}), 
а данные по H$_2$ и CO --- из обзора HERACLES 
(HERA CO-Line Extragalactic Survey; \citealp{Leroy+08}) 
и BIMA SONG 
(Berkeley-Illinois-Maryland Association Survey of Nearby Galaxies; 
\citealp{Helfer+03}). 
Для оценок темпа звёздообразования использовались наблюдения в далёком УФ 
диапазоне со спутника GALEX (Galaxy Evolution Explorer; 
\citealp{GildePaz+07}) 
и ИК наблюдения на длине волны 24 $\mu$m из обзора SINGS 
(the Spitzer Infrared Nearby Galaxies Survey; \citealp{Kennicutt+03}).
Профили поверхностной плотности звёзд определялись на основе данных SINGS 
на длине волны 3.6 $\mu$m. Кинематические данные (кривые вращения) 
брались из обзора THINGS.

Проведённое исследование не выявило ведущий процесс (из рассмотренных 
трёх), который бы полностью регулировал крупномасштабное 
звёздообразование. Но важным результатом является вывод, что при 
рассмотрении динамического статуса газового диска (устойчив он 
гравитационно или нет) влиянием звёзд (по схеме \citealp{JS84}) 
пренебрегать нельзя. Если это влияние учитывать, то для большей части 
рассмотренных галактик газовый диск оказывается лишь маржинально 
устойчивым со значение $Q_\mathrm{eff}=1.3-2.5$. При этом значение 
$Q_\mathrm{eff}$ лишь слабо возрастает по направлении к периферии в 
отличие от величины $Q_\mathrm{g}$, которая растёт быстро.

Работа \cite{Leroy+08} по-своему выдающаяся (статья занимает 64 
страницы), однако учёт кинематических эффектов в ней сделан не вполне 
корректно, а эффекты конечной толщины дисков вообще не учитывались. Да 
и использование гидродинамического приближения при рассмотрении 
звёздного диска нельзя считать оправданным. 

Что касается учёта кинематических эффектов, то замечания к этой 
процедуре сводятся к следующему. Для рассмотренной выборки галактик 
спектральные данные, из которых можно было бы получить профили 
дисперсии скоростей звёзд, вообще не рассматривались. Вместо этого 
предполагалось, что профиль дисперсии скоростей звёзд в радиальном 
направлении следует экспоненциальному закону
$$
\sigma_R(R) = \sigma_{R,0} \exp(-2h/R) \, ,
$$
где $h$ --- экспоненциальный масштаб звёздного диска, определяемый по 
поверхностной фотометрии. Нормировочный коэффициент находился из 
соотношения $\sigma_z = 0.6 \sigma_R$, величина $\sigma_z$ --- из 
условия гидростатического равновесия изотермического слоя с полутолщиной 
$z_0 = 2 h_z$
\be
\sigma_z^2 = \pi G z_0 \Sigma_\mathrm{s} \, ,
\label{equil_z}
\ee
а значение полутолщины бралось равным из отношения $2h/z_0 = 7.3 \pm 2.2$. 
Последнее отношение является средним значением для выборки галактик с ребра, 
изученной \cite{Kregel+02}. Использование таких усреднённых данных даёт 
для звёздных дисков выборки значение $Q_\mathrm{s} = 2-4$. И если 
нижняя оценка согласуется с предположением о маржинальной устойчивости 
звёздных дисков, то верхняя говорит скорее о динамической перегретости 
звёздных дисков, что может быть неверным.

Данные \cite{Leroy+08} были пересмотрены в работе \cite{RW11}. Здесь 
впервые была сделана попытка при рассмотрении двухжидкостной 
неустойчивости применительно к конкретным галактикам учесть эффект конечной 
толщины дисков по формуле (\ref{RW11z}). Сделан вывод, что значение 
эффективного параметра Тумре $Q_\mathrm{eff}$ в этом случае оказывается 
на 20-50\% больше, чем в случае бесконечно тонких дисков, т.е. 
стабилизирующим эффектом конечной толщины пренебрегать нельзя.

\section{Новые идеи}

\subsection{Объекты, для которых простой критерий не работает}

Существует ряд объектов и областей в них, для которых в областях 
текущего звёздообразования поверхностная плотность газа оказывается 
меньше критического (порогового) значения. На такие объекты (например, 
M33 и NGC~2043) указывалось ещё в первых работах по исследованию связи 
крупномасштабного звёздообразования с гравитационной неустойчивостью 
\citep{Kennicutt89,MK01}. К такого рода объектам, в первую очередь, 
относятся карликовые галактики (см., например, исследование 11 карликовых 
галактик в работе \citealp{vanZee+97}). На рисунках 12b,c из этой работы 
представлены азимутально усреднённые профили отношения 
$\Sigma_\mathrm{g} / \Sigma_\mathrm{g}^\mathrm{cr}$. Почти во всех 
случаях и на всём протяжении по радиусу диска это отношение оказывается 
меньше значения 0.67, найденного как пороговое значение в работе 
\cite{Kennicutt89}.

Периферийные области близких ярких галактик, в которых наблюдается 
звёздообразование, также имеют значения поверхностной плотности газа 
меньше порогового. Здесь примечательно исследование галактики NGC~6946 
\citep{Ferguson+98}. \cite{Ferguson+98} проанализировали 
$\mathrm{H}_{\alpha}$ изображения трёх близких галактик позднего типа 
(NGC~628, NGC~1058 и NGC~6946). Эти галактики демонстрируют присутствие 
областей HII в пределах двух оптических радиусов в полосе $B$. Как известно, 
наличие областей HII говорит о текущем звёздообразовании. 
При этом поверхностная яркость галактик в линии $\mathrm{H}_{\alpha}$ на 
периферии падает намного быстрее, чем поверхностная плотность областей 
HII. Отношение наблюдаемой поверхностной плотности газа HI к её критическому 
значению определено только для галактики NGC~6946, для которой есть 
хорошие кинематические данные. Согласно критерию~(\ref{kennicutt89}), 
$\Sigma_\mathrm{g} / \Sigma_\mathrm{g}^\mathrm{cr}$ оказывается больше 1 
в центральных областях, где звёздообразование наиболее интенсивно 
(большая поверхностная яркость в линии $\mathrm{H}_{\alpha}$), 
а газовый диск неустойчив. На периферии (за пределами одного 
оптического радиуса) газовый диск оказывается устойчивым, хотя области 
HII там видны, и их плотность практически не падает. Этот вывод сделан 
при тех предположениях, которые закладывались в анализ, проведённый в работе 
\cite{Kennicutt89}, т.е. дисперсия скоростей в газе (скорость звука) 
бралась постоянной и равной 6 км/с. Если учесть изменения скорости с 
расстоянием (как это получается из анализа профиля линии 21 см), то 
отношение $\Sigma_\mathrm{g} / \Sigma_\mathrm{g}^\mathrm{cr}$ 
оказывается меньше 0.67 на всём протяжении диска, несмотря на 
наблюдаемое звёздообразование.

Другим примером является галактика M83. В ней на основе данных УФ спутника 
GALEX обнаружены области 
звёздообразования на большом протяжении в диске галактики, включая 
периферию \citep{Thilker+05}. Хотя число областей звёздообразования во 
внешнем диске меньше, чем во внутреннем, они уверенно наблюдаются. 
Между тем, поверхностная плотность нейтрального водорода во внешних 
областях невелика и, по-видимому, меньше порогового значения для 
развития гравитационной неустойчивости \citep{Kennicutt89}. Такой же 
вывод был сделан относительно периферийных областей M31 
\citep{Cuillandre+01}, хотя особенностью звёздообразования здесь 
является формирование преимущественно звёзд малых масс.

Как уже отмечалось, \cite{Leroy+08} исследовали и другие процессы, 
которые могут регулировать крупномасштабное звёздообразование, не 
только гравитационную неустойчивость. В частности, они определили для 
галактик своей выборки профили отношения 
$\Sigma_\mathrm{S04}/\Sigma_\mathrm{g}$, где $\Sigma_\mathrm{S04}$ --- 
критическое значение поверхностной плотности газа, при которой в 
результате действия тепловой неустойчивости происходит образование 
холодных облаков и их дальнейшая молекуляризация \citep{Schaye04}. Для 
большей части галактик это отношение, как правило, меньше 1. Это 
говорит о том, что данный механизм работает. Но часто в периферийных 
областях $\Sigma_\mathrm{S04}/\Sigma_\mathrm{g} > 1$, и этот механизм 
не может служить альтернативой (или заменой) гравитационной неустойчивости 
в объяснении крупномасштабного звёздообразования.

На периферии галактик, а также в карликовых галактиках распределение 
водорода часто клочковатое. Фактор заполнения может колебаться от 6 до 
50\% \citep{Braun97}. Критерий \cite{Kennicutt89} не учитывает этот 
факт и оперирует с азимутально усреднённой поверхностной плотностью 
газа, тем самым занижая её значение в тех областях, где газа много и 
где могут быть сосредоточены области звёздообразования. Именно поэтому 
при неоднородном распределении газа правильнее картографировать области 
звёздообразования, отмечая локально (поскольку критерий неустойчивости 
локальный) те места, где $Q_\mathrm{g}$ (или $Q_\mathrm{eff}$) меньше~1, 
как это сделано в работе \cite{Yang+07} для БМО. Было бы интересно 
увидеть такой анализ для карликовых неправильных галактик и периферийных 
областей ярких галактик.

\subsection{Галактики с кольцами}

Ещё один интересный класс объектов, для которых анализ связи между 
крупномасштабным звёздообразованием и гравитационной неустойчивостью 
нужно проводить предельно корректно, это галактики с газовыми кольцами. 

Ряд таких галактик содержится в выборке \cite{Noordermeer+05}, 
объекты для которой были отобраны из обзора WHISP --- 
Westerbork HI survey of spiral and irregular galaxies 
\citep{Kamphius+96,vanderHulst+01}. Выборка \cite{Noordermeer+05} включает 
68 галактик с протяжёнными кривыми вращения и профилями поверхностной 
плотности HI --- $\Sigma_\mathrm{g}(R)$. Многие галактики показывают 
центральную депрессию поверхностной плотности, что может быть 
результатом плохого пространственного разрешения, но некоторые уверенно 
классифицируются как галактики, имеющие широкие пики поверхностностной 
плотности газа в промежуточных областях галактики.

Для 6 галактик из выборки, в которых газ сконцентрирован в протяжённых 
кольцах, \cite{Noordermeer+05} проанализировали связь между распределением 
нейтрального водорода и 
светимостью звёзд в полосе $B$, вклад в которую дают в основном молодые 
яркие звёзды. Особенно интересен анализ галактик UGC~3993, UGC~11914,  
UGC~2487 и UGC~6787. В первых двух галактиках на профилях яркости в 
полосе $B$ наблюдается уярчение в области газового кольца. Более того, 
у галактики UGC~11914 дополнительно обнаружено заметное излучение в линии 
$\mathrm{H}_{\alpha}$ 
как раз в кольце. Для галактики UGC~3993 таких данных нет, но звёзды в 
кольце более голубые, чем в центральных областях. Для двух других 
галактик в области кольца не наблюдаются ни уярчение в полосе $B$, ни 
эмиссия в линии $\mathrm{H}_{\alpha}$. 

Для всех этих галактик \cite{Noordermeer+05} построили профили 
отношения поверхностной плотности газа к пороговому значению плотности 
$\Sigma_\mathrm{g}/\Sigma_\mathrm{g}^\mathrm{cr89}$, 
согласно критерию~(\ref{kennicutt89}). Во всех случаях это отношение 
оказалось существенно меньше~1 (около 0.25), даже в области кольца. 
Это неудивительно для галактик UGC~2487 и UGC~6787, но кажется странным 
для галактик с активным звёздообразованием в кольце --- UGC~3993 и UGC~11914.

\cite{Silchenko+11} пересмотрели анализ данных для галактики UGC~11914, 
применив к ней критерий двухжидкостной неустойчивости согласно схеме 
\cite{Efstathiou00}, т.е. с учётом дестабилизирующего влияния звёздного 
диска. В результате отношение 
$\Sigma_\mathrm{g}/\Sigma_\mathrm{g}^\mathrm{cr,2}$ увеличилось до 
значения $0.5-0.6$ в области кольца, что всё равно мало для 
неустойчивого состояния. Следует, правда, иметь в виду следующее 
обстоятельство.

В работе \cite{Li+05} проанализированы результаты численных расчётов 
звёздообразования в самогравитирующих звёздных дисках в присутствии 
газа. Исследована гравитационная неустойчивость в таких моделях с точки 
зрения двухжидкостного критерия. Показано, что коллапс газовых масс 
происходит в тех областях, где $Q_\mathrm{eff} < 1.6$. Бурное 
звёздообразование наблюдается, когда $Q_\mathrm{eff} << 1$, но 
умеренный темп имеет место при уже $Q_\mathrm{eff} < 1.6$. 

Эти результаты означают, что при оценках отношения 
$\Sigma_\mathrm{g}/\Sigma_\mathrm{g}^\mathrm{cr,2}$ следует иметь в 
виду этот поправочный коэффициент (по сути означающий учёт 
неосесимметричных возмущений плотности). Тогда оценки величины 
$\Sigma_\mathrm{g}^\mathrm{cr,2}$ для галактики UGC~11914 
могут уменьшиться в 1.6 раза, и величина $\Sigma_\mathrm{g}$ в области 
кольца составит 80-90\% от порогового значения, что близко к состоянию 
маржинальной устойчивости.

\subsection{Двухжидкостный критерий с полными данными}

Примечательной особенностью работы \cite{Silchenko+11} является то, что 
при учёте дестабилизирующего влияния звёздного диска (см 
формулу~(\ref{2fluidJS1})) использовались реальные профили дисперсии 
скоростей звёзд в радиальном направлении --- $\sigma_R(R)$. Этого до 
сих пор не делалось ни в одной работе, изучающей связь 
крупномасштабного звёздообразования с критерием гравитационной 
неустойчивости.

В работе \cite{Silchenko+11} данные по звёздной кинематике были использованы 
для восстановления эллипсоида скоростей и радиальных профилей всех 
компонентов дисперсии скоростей звёзд. Более того, хотя галактика NGC~7217 
наблюдается под промежуточным углом наклона, были также определены профили 
толщины звёздного диска. Для этого использовался восстановленный профиль 
$\sigma_z(R)$ и условие равновесия звёздного диска в вертикальном 
направлении~(\ref{equil_z}).

Спектральные данные по абсорбционным линиям (звёздная кинематика) на 
протяжении в диске галактики в настоящее время имеются не более, чем 
для двух десятков объектов 
(например, 
\citealp{G+97,G+00,Sh+03,N+08,Zasov+08,Silchenko+11,GSh12,Zasov+12}).
Далеко не для всех этих объектов они использованы для восстановления 
эллипсоида скоростей, потому что процедура восстановления довольно сложна 
и относится к так называемым некорретным задачам.

Для галактик, наблюдаемых под промежуточными углами наклона $i$ к лучу 
зрения, компоненты дисперсии скоростей звёзд $\sigma_R$, $\sigma_{\varphi}$ 
и $\sigma_z$ связаны с наблюдаемой дисперсией скоростей на луче зрения вдоль 
большой и малой осей галактики, через следующие выражения:
\begin{equation}
\begin{array}{rcl}
\sigma_\mathrm{los,min}^2 (R \cos i) & = 
& \sigma_R^2 \, \sin^2 i + \sigma_z^2 \, cos^2 i \, ,\\
& &\\
\sigma_\mathrm{los,maj}^2 (R) & = 
& \sigma_\varphi^2 \, \sin^2 i + \sigma_z^2 \, cos^2 i \, .
\end{array}
\label{slos}
\end{equation}
Имея только два разреза, мы можем получить лишь линейную комбинацию  
$\sigma_R(R)$, $\sigma_\varphi(R)$ и $\sigma_z(R)$. Если звёздный диск 
находится в равновесии, то есть ещё одно соотношение, связывающее 
$\sigma_R(R)$ и $\sigma_\varphi(R)$ и среднюю азимутальную скорость 
звёзд $\bar{v}_\varphi$ \citep{BT87}
\begin{equation}
\frac{\sigma_\varphi^2}{\sigma_R^2} = \frac{1}{2} 
\left( 1 + \frac{\partial \ln \bar{v}_\varphi}{\partial \ln R}\right) 
\, .
\label{vphi}
\end{equation}
Если известна звёздная кривая вращения $\bar{v}_\varphi$ и хотя бы два 
разреза дисперсии скоростей звёзд вдоль луча зрения, то можно восстановить 
все кинематические данные.

Впервые эта схема была применена для определения эллипсоида скоростей 
для галактики NGC~488 в работе \cite{G+97}. К сожалению, схема 
восстановления даёт сильно зашумлённые данные, так как по сути она 
включает процедуру вычитания двух величин $\sigma_\mathrm{los,min}^2$ и 
$\sigma_\mathrm{los,maj}^2$, которые близки друг к другу. Ещё одно 
слабое место этой схемы --- численное дифференцирование профиля 
$\bar{v}_\varphi$. Чтобы избежать этих неприятных моментов, 
применяется параметризация находимых профилей дисперсии скоростей (см., 
например, \citealp{G+97,G+00,Sh+03}). Минус этого нововведения --- 
зависимость получаемых профилей от принятой параметризации.
Непараметрический подход к восстановлению эллипсоида скоростей 
предложен и осуществлён в работе \cite{N+08}. 

\cite{Silchenko+11} использовали аналогичную схему. Все кинематические 
профили, получаемые непосредственно из наблюдений, аппроксиморовались 
полиномами, а затем восстанавливаемые профили получались с 
использованием этих аналитических приближений. Правда, даже в этом 
случае вычитание профилей дисперсии скоростей вдоль большой и малой 
осей приводит к неустойчивым решениям. Поэтому в работе \cite{Silchenko+11} 
для восстановления профиля $\sigma_R$ использовалось уравнение для 
асимметричного сдвига \citep{BT87}: 
\begin{equation}
v_\mathrm{c}^2 - \bar{v}_\varphi^2 = 
\sigma_R^2 \left( 
\frac{\sigma_\varphi^2}{\sigma_R^2} -1 - 
\frac{\partial \ln \Sigma_\mathrm{s}}{\partial \ln R} - 
\frac{\partial \ln \sigma_R^2}{\partial \ln R}
\right) \, ,
\label{AsDr}
\end{equation}
Если отношение массы к светимости принять постоянной вдоль радиуса 
величиной, то в уравнении~(\ref{AsDr}) можно использовать поверхностную 
яркость звёздного диска вместо поверхностной плотности 
$\Sigma_\mathrm{s}$ в тех полосах, которые трассируют старое звёздное 
население.

Таким образом, восстановив эллипсоид скоростей, мы получаем возможность 
применить критерий двухжидкостной неустойчивости, не делая никаких 
предположений относительно профиля $\sigma_R$. Более того, зная профиль 
$\sigma_z$, мы можем корректно учесть эффект конечной толщины звёздного 
диска, не пребегая к приближению \cite{RW11}. Необходимость корректного 
учёта эффекта конечной толщины диктуется ещё тем, что величина 
анизотропии скоростей $\sigma_z / \sigma_R$ варьируется в щироких 
пределах для галактик разных типов --- от $0.25$ (поздние типы) до $0.75$ 
для ранних типов \citep{G+97,G+00,Sh+03,GSh12}). Согласно 
\cite{Silchenko+11} это отношение также сильно варьируется с радиусом в 
пределах одной галактики, поэтому приближение \cite{RW11}, в котором 
эффект толщины учитывается через постоянное отношение 
$\sigma_z / \sigma_R$, равное 0.6, представляется слишком грубым.

Таким образом, для тех галактик, для которых имеются данные по 
кинематике и распределению HI, а также кинематические данные для звёзд, 
имеется возможность исследовать динамический статус их газовых дисков 
предельно корректно. Это особенно важно для тех объектов, которые 
показывают области интенсивного звёздообразования (например, в газовых 
кольцах), но простое применение критерия гравитационной неустойчивости 
давало устойчивое состояние их газовых дисков. 

\begin{thebibliography}{99}

   \bibitem[{{Bertin} \& {Romeo}(1988)}]{BR88}
    {Bertin}, G., {Romeo}, A.~B. 1988, A\&A, 195, 105

   \bibitem[{{Binney} \& {Tremaine}(1987)}]{BT87}
    {Binney}, J. \& {Tremaine}, S. 1987, Galactic Dynamics, 
    (Princeton, NY, Princeton University Press)

   \bibitem[{{Boissier et al.}(2003)}]{Boissier+03}
    {Boissier}, S., {Prantzos}, N., {Boselli}, A., 
    {Gavazzi} G. 2003, MNRAS, 346, 1215

   \bibitem[{{Bottema}(1993)}]{Bottema93}
    {Bottema}, R. 1993, A\&A, 275, 16

   \bibitem[{{Braun}(1997)}]{Braun97}
    {Braun}, R. 1997, ApJ, 484, 637

   \bibitem[{{Cuillandre et al.}(2001)}]{Cuillandre+01}
    {Cuillandre}, J., {Lequeux}, J., {Allen}, R.~J., {Mellier}, Y., 
    {Bertin}, E. 2001, ApJ, 554, 190
  
   \bibitem[{{Dehnen} \& {Binney}(1998)}]{DB98}
    {Dehnen} W., {Binney} J.~J., 1998, MNRAS, 298, 387
  
   \bibitem[{{Efstathiou}(2000)}]{Efstathiou00}
    {Efstathiou}, G. 2000, MNRAS, 317, 697

   \bibitem[{{Elmegreen}(1995)}]{Elmegreen95}
    {Elmegreen}, B.~G. 1995, MNRAS, 275, 944

   \bibitem[{{Helfer et al.}(2003)}]{Helfer+03}
    {Helfer}, T.~T., {Thornley}, M.~D., {Regan}, M.~W., {Wong}, T., 
    {Sheth}, K., {Vogel}, S.~N., {Blitz}, L., {Bock}, D.~C.-J. 
    2003, ApJS, 145, 259

   \bibitem[{{Ferguson et al.}(1998)}]{Ferguson+98}
    {Ferguson}, A.~M.~N., {Wyse}, R.~F.~G., {Gallagher}, J.~S., 
    {Hunter} D.~A. 1998, ApJ, 506, L19

   \bibitem[{{Gerssen et al.}(1997)}]{G+97}
    {Gerssen} J., {Kuijken} K., {Merrifield} M. 1997, MNRAS, 288, 618

   \bibitem[{{Gerssen et al.}(2000)}]{G+00}
    {Gerssen} J., {Kuijken} K., {Merrifield} M. 2000, MNRAS, 317, 545

   \bibitem[{{Gerssen} \& {Shapiro Griffin}(2012)}]{GSh12}
    {Gerssen} J., {Shapiro Griffin} K. 2012, MNRAS, 423, 2726

   \bibitem[{{Gil de Paz et al.}(2007)}]{GildePaz+07}
    {Gil de Paz}, A., et al. 2007, ApJS, 173, 185

   \bibitem[{{Goldreich} \& {Linden-Bell}(1965)}]{GL_B65}
    {Goldreich}, P. \& {Linden-Bell}, D. 1965, MNRAS, 130, 97

   \bibitem[{{Jog}(1996)}]{Jog96}
    {Jog}, C.~J. 1996, MNRAS, 278, 209

   \bibitem[{{Jog} \& {Solomon}(1984)}]{JS84}
    {Jog}, C.~J. \& {Solomon}, P.~M. 1984, ApJ, 276, 114

   \bibitem[{{Kamphius} {et~al.}(1996)}]{Kamphius+96}
   {Kamphius}, J.~J., {Sijbring}, D., {van Albada}, T.~S. 1996,
   A\&AS, 116, 15

   \bibitem[{{Kennicutt}(1989)}]{Kennicutt89}
    {Kennicutt}, Jr., R.~C. 1989, ApJ, 344, 685

   \bibitem[{{Kennicutt et al.}(2003)}]{Kennicutt+03}
    {Kennicutt}, Jr., R.~C., Jr., et al. 2003, PASP, 115, 928

   \bibitem[{{Kregel et al.}(2002)}]{Kregel+02}
    {Kregel}, M., {van der Kruit}, P.~C., {de Grijs}, R. 
    2002, MNRAS, 334, 646

   \bibitem[{{Li et al.}(2005)}]{Li+05}
    {Li}, Y., {Mac Low} M.-M., {Klessen} R.~S. 2005, ApJ, 626, 823

   \bibitem[{{Leroy et al.}(2008)}]{Leroy+08}
    {Leroy}, A.~K., {Walter}, F., {Brinks}, E., 
    {de Block} W.~J.~G., {Madore} B., {Thornley} M.~D. 
    2008, AJ, 136, 2782

   \bibitem[{{Martin} \& {Kennicutt}(2001)}]{MK01}
    {Martin}, K.~L., {Kennicutt}, Jr., R.~C. 2001, ApJ, 555, 301

   \bibitem[{{Noordermeer} {et~al.}(2005)}]{Noordermeer+05}
   {Noordermeer}, E., {van der Hulst}, J.~M., {Sancisi}, R., 
   {Swaters}, R.~A., \& {van Albada}, T.~S. 2005, A\&A, 442, 137

   \bibitem[{{Noordermeer et~al.}(2008)}]{N+08}
    {Noordermeer}, E., {Merrifield}, M.~R., 
    {Arag{\'o}n-Salamanca}, A. 2008, MNRAS, 388, 1381

   \bibitem[{{Rafikov}(2001)}]{Rafikov01}
    {Rafikov}, R.~R. 2001, MNRAS, 323, 445

   \bibitem[{{Romeo}(1992)}]{Romeo92}
    {Romeo}, A.~B. 1992, MNRAS, 256, 307

   \bibitem[{{Romeo}(1994)}]{Romeo94}
    {Romeo}, A.~B. 1994, A\&A, 286, 799

   \bibitem[{{Romeo} \& {Weigert}(2011)}]{RW11}
    {Romeo}, A.~B., {Weigert}, J. 2011, MNRAS, 416, 1191

   \bibitem[{{Shapiro et al.}(2012)}]{Sh+03}
    {Shapiro}, K.~L., {Gerssen} J., {van der Marel} R.~P. 
    2003, AJ, 126, 2707
    
   \bibitem[{{Schaye}(2004)}]{Schaye04}
    {Schaye}, J. 2004, ApJ, 609, 667

   \bibitem[{{Schmidt}(1959)}]{Schmidt59}
    {Schmidt}, M. 1959, ApJ, 129, 243

   \bibitem[{{Sil'chenko} {et~al.}(2011)}]{Silchenko+11}
   {Sil'chenko}, O.~K., {Chilingarian}, I.~V., {Sotnikova}, N.~Ya., 
   {Afanasiev}, V.~L. 2011, MNRAS, 414, 364

   \bibitem[{{Thilker}(2005)}]{Thilker+05}
    {Thilker}, D.~A., {Bianchi}, L., {Boissier}, S., et al. 
    2005, ApJ, 619, L79

   \bibitem[{{Toomre}(1964)}]{Toomre64}
    {Toomre}, A. 1964, ApJ, 139, 1217

   \bibitem[{{van der Hulst} {et~al.}(2001)}]{vanderHulst+01}
   {van der Hulst}, J.~M., {van Albada}, T.~S. {Sancisi}, R. 2001,
   in Gas and Galaxy Evolution, ASP Conf. Ser. 240, 451

   \bibitem[{{van Zee et al.}(1997)}]{vanZee+97}
    {van Zee}, L., {Haynes}, M.~P., {Salzer}, J.~J., 
    {Broeils}, A.~H. 1997, AJ, 113, 1618

   \bibitem[{{Walter et al.}(2008)}]{Walter+08}
    {Walter} F. et al. 2008, AJ, 136, 2563	       

   \bibitem[{{Wang} \& {Silk}(1994)}]{WS94}
    {Wang}, B., {Silk}, J. 1994, ApJ, 427, 759

   \bibitem[{{Yang et al.}(2007)}]{Yang+07}
    {Yang}, C.-C., {Gruendl}, R.~A., {Chu}, Y.-H., 
    {Mac Low} M.-M., {Fukui} Y. 2007, ApJ, 671, 374

   \bibitem[{{Zasov et al.}(2008)}]{Zasov+08}
    {Zasov}, A.~V., {Moiseev}, A.,V., {Khoperskov}, A.V., 
    {Smirnova}, E.~A. 2008, Astronomy Reports, 52, 79
    
   \bibitem[{{Zasov et al.}(2012)}]{Zasov+12}
    {Zasov}, A.~V., {Khoperskov}, A.V., {Katkov}, I.~Y., 
    {Afanasiev}, V.~L., {Kaisin}, S.~S. 2012, 
    Astrophysical Bulletin, 67, 362

\end{thebibliography}

\end{document}
