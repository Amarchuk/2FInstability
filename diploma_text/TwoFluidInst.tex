%\documentclass[a4paper,12pt]{article}
\documentclass[russian,12pt]{article}

\usepackage[utf8]{inputenc}
\usepackage[russian]{babel}
\usepackage[]{natbib}
%\usepackage{psfig,amsfonts}

\textheight 23.0cm
\textwidth 16.5cm

\hoffset=-1.25cm
\voffset=-1.5cm
\baselineskip=20pt
\sloppy

\def\bs{\bigskip}
\def\noi{\noindent}

\def\bc{\begin{center}}
\def\ec{\end{center}}
\def\be{\begin{equation}}
\def\ee{\end{equation}}

\begin{document}

\bc
\Large\bf 
Двухжидкостная неустойчивость и крупномасштабное звёздообразование в 
галактиках
\ec

\section{Аннотация}

Будет исследована устойчивость газового диска конкретных спиральных 
галактик на основе критерия двухжидкостной неустойчивости 
(\citealt{JS84,Efstathiou00}) и определены профили критической поверхностной 
плотности газа. Эти профили будут сравниваться с наблюдаемым распределением 
газа HI$+$H$_2$. Условия наступления неустойчивости будут проанализированы 
с точки зрения идущего в галактиках крупномасштабного звёздообразования, 
индикатором которого является излучение в линии H$_\alpha$. Особое внимание 
будет уделено галактикам, у которых излучение в линии H$_\alpha$ наблюдается, 
а газовый диск в областях звёздообразования оказывается устойчивым согласно 
стандартному критерию гравитационной неустойчивости 
(\citealt{Toomre64,GL_B65,Kennicutt89}). 
Возможно, решение этой загадки будет найдено в рамках критерия 
двухжидкостной неустойчивости.

\section{Описание задачи}

Для решения поставленной задачи будет отобран ряд галактик с известными 
кривыми вращения как по линиям HI (газовые кривые вращения --- 
$v_\mathrm{c}(R)$), так и по абсорбционным линиям (звёздные кривые 
вращения --- $v_\varphi(R)$). За основу будет взята выборка 
\cite{Noordermeer+05}, объекты для которой были отобраны из обзора WHISP --- 
Westerbork HI survey of spiral and irregular galaxies 
\citep{Kamphius+96,vanderHulst+01}. Выборка \cite{Noordermeer+05} включает 
68 галактик с протяжёнными кривыми вращения и профилями поверхностной 
плотности HI --- $\Sigma_\mathrm{g}(R)$. 
Данные по звёздной кинематике будут найдены в литературе.

Кривые вращения совместно с известными по литературе профилями 
дисперсии лучевых скоростей $\sigma_\mathrm{los}(R)$ дадут возможность 
восстановить форму эллипсоида скоростей и профиль дисперсии скоростей в 
радиальном направлении $\sigma_\mathrm{R}$. Процедура восстановления 
описана в работе \cite{Sil'chenko+11} и сводится к определению 
профиля $\sigma_\mathrm{R}$ через известное соотношение, называемое 
асимметричным сдвигом %\citep{BT87}: 
\begin{equation}
v_\mathrm{c}^2 - v_\varphi^2 = 
\sigma_R^2 \left( 
\frac{\sigma_\varphi^2}{\sigma_R^2} -1 - 
\frac{\partial \ln \Sigma_\mathrm{d}}{\partial \ln R} - 
\frac{\partial \ln \sigma_R^2}{\partial \ln R}
\right) \, ,
\label{AsDr}
\end{equation}
где $\Sigma_\mathrm{d}$ --- поверхностная плотность звёзд диска, а 
$\sigma_\varphi$ --- дисперсия скоростей звёзд в азимутальном 
направлении. Для диска, находящегося в равновесии,
$$
\frac{\sigma_\varphi^2}{\sigma_R^2} = \frac{1}{2} 
\left( 1 + \frac{\partial \ln v_\varphi}{\partial \ln R}\right) 
\, .
$$

Кроме этого, для отобранных галактик будут собраны данные по 
декомпозиции изображений галактик на балдж и диск (за основу будут взяты 
параметры декомпозиции из работы \cite{NvdH07}). Тем самым будут 
определены профили поверхностной яркости диска $I_\mathrm{d}(R)$. В тех 
случаях, когда фотометрические параметры диска и балджа не будут 
найдены в литературе, декомпозиция будет сделана самостоятельно при 
помощи пакета Galfit. %\citep{}.

Если фотометрия известна хотя бы для двух полос, можно 
перейти от поверхностной яркости $I_\mathrm{d}(R)$ 
к поверхностной плотности звёзд диска через откалиброванное отношение 
массы к светимости для исследуемой полосы $({\cal{M}}/L)_\mathrm{X}$, 
оцененное по цветам \citep{Bell+03}
$$
\Sigma_\mathrm{d}(R) = ({\cal{M}}/L)_\mathrm{X} 
I_\mathrm{d}^\mathrm{X}(R) \, .
$$

Данные, взятые вместе --- 
кривая вращения $v_\mathrm{c}(R)$, 
профили дисперсии скоростей звёзд $\sigma_\mathrm{R}(R)$
и поверхностных плотностей газового и звёздного дисков 
$\Sigma_\mathrm{g}(R)$ и $\Sigma_\mathrm{d}(R)$, --- 
позволяют найти профиль критической поверхностной плотности газа, 
превышение которой делает диск неустойчивым (\citealt{JS84,Efstathiou00})
\begin{equation}
\Sigma_\mathrm{cr,2} = \frac{\kappa c}{3.36 G g(a,b)} \, ,
\label{2fluid}
\end{equation}
где $\kappa$ --- эпициклическая частота, определяемая по кривой 
вращения
$$
\displaystyle \kappa = 
\sqrt{2 \left(\frac{v_{\rm c}^2}{R^2} + 
\frac{v_{\rm c}}{R} \frac{d v_{\rm c}}{dR} \right)} \, ,
$$
$c$ --- скорость звука в газе (или дисперсия скоростей газовых облаков), 
которую можно считать постоянной и равной 6 км/с, 
$a = \sigma_\mathrm{R}/c $ --- отношение дисперсии скоростей звёзд к 
дисперсии скоростей в газе, 
$b = \Sigma_\mathrm{d}/\Sigma_\mathrm{g}$ --- отношение поверхностных 
плотностей звёздного и газового дисков, 
а $g(a,b)$ --- функция, численно рассчитанная в работе \citet{Efstathiou00}.

Особое внимание будет уделено галактикам, у которых распределение 
нейтрального водорода демонстрирует кольцо, и тем галактикам, у которых 
в области кольца наблюдается интенсивное звёздообразование.

Предполагается

\begin{enumerate}

\item[1)]
сравнив профили критической поверхностной плотностей газа, 
определённые согласно критерию двухжидкостной неустойчивости 
(\ref{2fluid}) и простому одножидкостному критерию \cite{GL_B65} 
(аналог критерия \citep{Toomre64} для газового диска)
\begin{equation}
\Sigma_\mathrm{cr,1} = \frac{\kappa c}{3.36 G} \, ,
\label{1fluid}
\end{equation}
сделать выводы о реальном динамическом статусе газовых дисков 
конкретных галактик;

\item[2)]
отобрать галактики, для которых переооценка запаса прочности газового 
диска на основе одножидкостного критерия, наиболее существенна;

\item[3)]
для этих галактик сопоставить области звёздообразования (профили яркости в 
линии H$_\alpha$) с областями, где газовый диск неустойчив по критерию 
двухжидкостной неустойчивости;

\item[4)]
уточнить выводы о роли гравитационной неустойчивости в регулировании 
крупномасштабного звёздообразования.

\end{enumerate}

\begin{thebibliography}{99}

   \bibitem[{{Bell} {et~al.}(2003){Bell}, {McIntosh},
   {Katz}, \& {Weinberg}}]{Bell+03}
   {Bell}, E.~F., {McIntosh}, D.~H., {Katz}, N., 
   \& {Weinberg}, M.~D. 2003, ApJS, 149, 289

   \bibitem[{{Efstathiou}(2000)}]{Efstathiou00}
   {Efstathiou}, G. 2000, MNRAS, 317, 697

   \bibitem[{{Goldreich} \& {Linden-Bell}(1965)}]{GL_B65}
    {Goldreich}, P. \& {Linden-Bell}, D. 1965, MNRAS, 130, 97

   \bibitem[{{Jog} \& {Solomon}(1984)}]{JS84}
    {Jog}, C.~J. \& {Solomon}, P.~M. 1984, ApJ, 276, 114

   \bibitem[{{Kennicutt}(1989)}]{Kennicutt89}
   {Kennicutt}, Jr., R.~C. 1989, ApJ, 344, 685
   
   \bibitem[{{Kamphius} {et~al.}(1996){Kamphius}, {Sijbring}, \&
   {van Albada}}]{Kamphius+96}
   {Kamphius}, J.~J., {Sijbring}, D., \& {van Albada}, T.~S. 1996,
   A\&AS, 116, 15
   
   \bibitem[{{Noordermeer} {et~al.}(2008){Noordermeer}, {Merrifield}, \&
   {Arag{\'o}n-Salamanca}}]{NMA08}
   {Noordermeer}, E., {Merrifield}, M.~R., \& {Arag{\'o}n-Salamanca}, A. 2008,
   MNRAS, 388, 1381

   \bibitem[{{Noordermeer} \& {van der Hulst}(2007)}]{NvdH07}
   {Noordermeer}, E. \& {van der Hulst}, J.~M. 2007, MNRAS, 376, 1480

   \bibitem[{{Noordermeer} {et~al.}(2005){Noordermeer}, {van der Hulst},
   {Sancisi}, {Swaters}, \& {van Albada}}]{Noordermeer+05}
   {Noordermeer}, E., {van der Hulst}, J.~M., {Sancisi}, R., 
   {Swaters}, R.~A., \& {van Albada}, T.~S. 2005, A\&A, 442, 137

   \bibitem[{{Sil'chenko} {et~al.}(2011){Sil'chenko}, {Chilingarian},
   {Sotnikova}, \& {Afanasiev}}]{Sil'chenko+11}
   {Sil'chenko}, O.~K., {Chilingarian}, I.~V., {Sotnikova}, N.~Ya., 
   \& {Afanasiev}, V.~L. 2011, MNRAS, 414, 364

   \bibitem[{{Toomre}(1964)}]{Toomre64}
   {Toomre}, A. 1964, ApJ, 139, 1217

   \bibitem[{{van der Hulst} {et~al.}(2001){van der Hulst}, {van Albada}, \&
   {Sancisi}}]{vanderHulst+01}
   {van der Hulst}, J.~M., {van Albada}, T.~S. \& {Sancisi}, R. 2001,
   in Gas and Galaxy Evolution, ASP Conf. Ser. 240, 451
	       
\end{thebibliography}

\end{document}
