%\documentclass[a4paper,10pt]{article}
\documentclass[english,10pt]{article}

\usepackage[utf8]{inputenc}
\usepackage[russian,english]{babel}
%\usepackage{psfig,amsfonts}
\usepackage{array,longtable}
\usepackage{amssymb}
\usepackage{graphicx} 
\usepackage{graphics}
\setcounter{LTchunksize}{20}
\textheight 23.0cm
\textwidth 18.0cm

\hoffset=-2.5cm
%\voffset=-2.5cm
\baselineskip=22pt
\sloppy

\renewcommand{\refname}{ }
\renewcommand{\tablename}{Таблица}
\renewcommand{\figurename}{Рис.}

\def\bc{\begin{center}}
\def\ec{\end{center}}

\def\lb{\linebreak}
\def\be{\begin{equation}}
\def\ee{\end{equation}}
\def\beq{\begin{eqnarray}}
\def\eeq{\end{eqnarray}}
\def\bfig{\begin{figure}}
\def\efig{\end{figure}}
\def\bnum{\begin{enumerate}}
\def\enum{\end{enumerate}}
\def\ds{\displaystyle}
\def\mm{\mathrm}

\begin{document}

\noindent
Noordermeer E., van der Hulst J. M., 
``The stellar mass distribution in early-type disc galaxies: surface
photometry and bulge-disc decompositions'', 
MNRAS, 376, 1480-1512 (2007)

%*******************************************************DISK********************
\begin{longtable}[c]{llllllllllllll}
\caption{NGC 2273. Structural parameters of the galaxy}\label{GAL}  \\ 
\hline 
band & scale & 
$\mu_\mm{e,b}$ & $\mu_\mm{e,b}^\mm{c}$ & $r_\mm{e,b}$ & 
$n$ & $m_\mm{b}$ & $M_\mm{b}$ & 
$\mu_\mm{0,d}$ & $\mu_\mm{0,d}^\mm{c}$ & $h$ & 
$m_\mm{d}$ & $M_\mm{d}$ & $B/D$ \\ 
& (kpc/$''$) & \multicolumn{2}{c}{(mag arcsec$^{-2}$)} & (arcsec) 
& & \multicolumn{2}{c}{(mag)} & 
\multicolumn{2}{c}{(mag arcsec$^{-2}$)} & & 
\multicolumn{2}{c}{(mag)} & \\
(1)&(2)&(3)&(4)&(5)&(6)&(7)&(8)&(9)&(10)&(11)&(12)&(13)&(14) \\ 
\hline
\endfirsthead 
\hline
\multicolumn{14}{c}{\small\slshape Structural parameters of the galaxy. 
Continued. } \\ \hline
band & scale & 
$\mu_\mm{e,b}$ & $\mu_\mm{e,b}^\mm{c}$ & $r_\mm{e,b}$ & 
$n$ & $m_\mm{b}$ & $M_\mm{b}$ & 
$\mu_\mm{0,d}$ & $\mu_\mm{0,d}^\mm{c}$ & $h$ & 
$m_\mm{d}$ & $M_\mm{d}$ & $B/D$ \\ 
& (kpc/$''$) & \multicolumn{2}{c}{(mag arcsec$^{-2}$)} & (arcsec) 
& & \multicolumn{2}{c}{(mag)} & 
\multicolumn{2}{c}{(mag arcsec$^{-2}$)} & & 
\multicolumn{2}{c}{(mag)} & \\
(1)&(2)&(3)&(4)&(5)&(6)&(7)&(8)&(9)&(10)&(11)&(12)&(13)&(14) \\
\hline
\endhead 
\hline
$R$ & 0.13 & 17.48 & 17.29 & 2.2 & 1.0 & 13.40 & -18.97 & 
19.15 & 19.49 & 21.1 & 11.13 & -21.24 & 0.12 \tabularnewline

$B$ & 0.13 & 19.14 & 18.84 & 2.2 & 1.0 & 15.07 & -17.42 & 
20.71 & 20.93 & 22.0 & 12.56 & -19.93 & 0.10 \tabularnewline
\hline
\end{longtable}

Columns: 
(1) Photometric band. 
(2) Conversion factor to convert arcsecs into kpc.
(3) Bulge effective surface brightness. 
(4) Idem, but corrected for galactic foreground extinction.
(5) Effective radius of the bulge, given in arcsec.
(6) S\`ersic index. 
(7) Bulge total apparent magnitude.
(8) Bulge total absolute magnitude.
(9) Disc central surface brightness. 
(10) Idem, but corrected for galactic foreground extinction.
(11) Disc scalelength, given in arcsec.
(12) Disc total apparent magnitude.
(13) Disc total absolute magnitude. 
(14) The ratio of the bulge to disc luminosities. 

\bigskip
\noindent
Noordermeer E., van der Hulst J.M., Sancisi R., 
Swaters R. S., and van Albada T.S., 
``The mass distribution in early-type disc galaxies: declining rotation
curves and correlations with optical properties'', 
MNRAS, 376, 1513-1546 (2007)

%*******************************************************DISK********************
\begin{longtable}[c]{cccccccccc}
\caption{NGC 2273. Basic data}\label{GAL}  \\ 
\hline 
Type & D & $M_B$ & $M_R$ & $\mu_\mm{0,d}^\mm{c}$ & $h$ & $r_\mm{e,b}$
& $V_\mm{sys}$ & $PA$ & $i$ \\ 
& (Mpc) & (mag) & (mag) & (mag arcsec$^{-2}$) & (kpc) & (kpc)
& (km/s) & (deg) & (deg) \\
(1)&(2)&(3)&(4)&(5)&(6)&(7)&(8)&(9)&(10) \\ 
\hline
\endfirsthead 
\hline
\multicolumn{6}{c}{\small\slshape Basic data. 
Continued. } \\ \hline
Type & D & $M_B$ & $M_R$ & $\mu_\mm{0,d}^\mm{c}$ & $h$ & $r_\mm{e,b}$
& $V_\mm{sys}$ & $PA$ & $i$ \\ 
& (Mpc) & (mag) & (mag) & (mag arcsec$^{-2}$) & (kpc) & (kpc)
& (km/s) & (deg) & (deg) \\
(1)&(2)&(3)&(4)&(5)&(6)&(7)&(8)&(9)&(10) \\ 
\hline
\endhead 
\hline
SB(r)a & 27.3 & -20.02 & -21.35 & 19.49 & 2.8 & 0.3 
& 1837 & 56 & 55 \tabularnewline
\hline
\end{longtable}

Columns: 
(1) Morphological type from NED). 
(2) Distance. 
(3), (4) absolute B-and R-band magnitudes 
(corrected for Galactic foreground extinction).
(5) R-band central disc surface brightness 
(corrected for Galactic foreground extinction and inclination effects).
(6) $R$-band disc scalelength.
(7) $R$-band bulge effective radius.
(8) Heliocentric systemic velocity.
(9) Position angle (north through east) of major axis. 
(10)Inclination angle.

\bigskip
\noindent
Фотометрия в $B$ и $R$ даёт согласованные значения центральной 
поверхностной плотности для диска. Однако, приведённая фотометрия 
не учитывает наличие бара. По-видимому, внешний диск имеет 
б\'{о}льшую шкалу. Декомпозиция галактики противоречивая. Разные авторы 
приводят разные значения параметров. Но диск, скорее всего, имеет 
б\'{о}льшую шкалу.

\bigskip
\noindent
Галактика с баром раннего типа. Бар, по-видимому, двойной. 
Сейфертовская галактика (Mrk 620). Центральные области много изучались. 
В диске хорошо видны два спиральных рукава. В SDSS изображения нет. 
Излучение $H_\alpha$ видно везде. Довольно мощное в области бара. 
В диске узлы разбросаны вдоль спиральных рукавов. Внешний бар 
окружён кольцом водорода (примерно на 100 arcsec). Здесь же 
голубые звёзды. На профиле яркости в полосе $B$ в этой области 
уярчение.

\bigskip
\noindent
У галактики умеренное количество газа. Интегрально $M_\mm{HI} = 1.65 \cdot 
10^{9} M_\odot$. Но газ сосредоточен в области газового кольца примерно на 
90-100 arcsec. 

\bigskip
\noindent
Газовая (WSRT) и звёздная кинематика плохо согласуются (точки на 
звёздной кривой вращения лежат выше точек на газовой кривой 
вращения). Профиль дисперсии скоростей $\sigma_R$ восстанавливать 
через $\sigma_\mm{los}$.

\end{document}
