%\documentclass[a4paper,10pt]{article}
\documentclass[english,10pt]{article}

\usepackage[utf8]{inputenc}
\usepackage[russian,english]{babel}
%\usepackage{psfig,amsfonts}
\usepackage{array,longtable}
\usepackage{amssymb}
\usepackage{graphicx} 
\usepackage{graphics}
\setcounter{LTchunksize}{20}
\textheight 23.0cm
\textwidth 18.0cm

\hoffset=-2.5cm
%\voffset=-2.5cm
\baselineskip=22pt
\sloppy

\renewcommand{\refname}{ }
\renewcommand{\tablename}{Таблица}
\renewcommand{\figurename}{Рис.}

\def\bc{\begin{center}}
\def\ec{\end{center}}

\def\lb{\linebreak}
\def\be{\begin{equation}}
\def\ee{\end{equation}}
\def\beq{\begin{eqnarray}}
\def\eeq{\end{eqnarray}}
\def\bfig{\begin{figure}}
\def\efig{\end{figure}}
\def\bnum{\begin{enumerate}}
\def\enum{\end{enumerate}}
\def\ds{\displaystyle}
\def\mm{\mathrm}

\begin{document}

\noindent
Noordermeer E., van der Hulst J. M., 
``The stellar mass distribution in early-type disc galaxies: surface
photometry and bulge-disc decompositions'', 
MNRAS, 376, 1480-1512 (2007)

%*******************************************************DISK********************
\begin{longtable}[c]{llllllllllllll}
\caption{NGC 338. Structural parameters of the galaxy}\label{GAL}  \\ 
\hline 
band & scale & 
$\mu_\mm{e,b}$ & $\mu_\mm{e,b}^\mm{c}$ & $r_\mm{e,b}$ & 
$n$ & $m_\mm{b}$ & $M_\mm{b}$ & 
$\mu_\mm{0,d}$ & $\mu_\mm{0,d}^\mm{c}$ & $h$ & 
$m_\mm{d}$ & $M_\mm{d}$ & $B/D$ \\ 
& (kpc/$''$) & \multicolumn{2}{c}{(mag arcsec$^{-2}$)} & (arcsec) 
& & \multicolumn{2}{c}{(mag)} & 
\multicolumn{2}{c}{(mag arcsec$^{-2}$)} & & 
\multicolumn{2}{c}{(mag)} & \\
(1)&(2)&(3)&(4)&(5)&(6)&(7)&(8)&(9)&(10)&(11)&(12)&(13)&(14) \\ 
\hline
\endfirsthead 
\hline
\multicolumn{14}{c}{\small\slshape Structural parameters of the galaxy. 
Continued. } \\ \hline
band & scale & 
$\mu_\mm{e,b}$ & $\mu_\mm{e,b}^\mm{c}$ & $r_\mm{e,b}$ & 
$n$ & $m_\mm{b}$ & $M_\mm{b}$ & 
$\mu_\mm{0,d}$ & $\mu_\mm{0,d}^\mm{c}$ & $h$ & 
$m_\mm{d}$ & $M_\mm{d}$ & $B/D$ \\ 
& (kpc/$''$) & \multicolumn{2}{c}{(mag arcsec$^{-2}$)} & (arcsec) 
& & \multicolumn{2}{c}{(mag)} & 
\multicolumn{2}{c}{(mag arcsec$^{-2}$)} & & 
\multicolumn{2}{c}{(mag)} & \\
(1)&(2)&(3)&(4)&(5)&(6)&(7)&(8)&(9)&(10)&(11)&(12)&(13)&(14) \\
\hline
\endhead 
\hline
$R$ & 0.32 & 21.10 & 20.95 & 15.0 & 3.7 & 12.30 & -21.91 & 
21.17 & 21.92 & 18.3 & 13.34 & -20.87 & 2.60 \tabularnewline

$B$ & 0.32 & 22.94 & 22.70 & 15.0 & 3.7 & 14.19 & -20.12 & 
21.87 & 22.53 & 17.7 & 14.34 & -19.97 & 1.15 \tabularnewline

$I$ & 0.32 & 20.52 & 20.41 & 15.0 & 3.7 & 11.82 & -22.35 & 
19.00 & 19.79 & 12.9 & 12.26 & -21.91 & 1.50 \tabularnewline
\hline
\end{longtable}

Columns: 
(1) Photometric band. 
(2) Conversion factor to convert arcsecs into kpc.
(3) Bulge effective surface brightness. 
(4) Idem, but corrected for galactic foreground extinction.
(5) Effective radius of the bulge, given in arcsec.
(6) S\`ersic index. 
(7) Bulge total apparent magnitude.
(8) Bulge total absolute magnitude.
(9) Disc central surface brightness. 
(10) Idem, but corrected for galactic foreground extinction.
(11) Disc scalelength, given in arcsec.
(12) Disc total apparent magnitude.
(13) Disc total absolute magnitude. 
(14) The ratio of the bulge to disc luminosities. 

\bigskip
\noindent
Noordermeer E., van der Hulst J.M., Sancisi R., 
Swaters R. S., and van Albada T.S., 
``The mass distribution in early-type disc galaxies: declining rotation
curves and correlations with optical properties'', 
MNRAS, 376, 1513-1546 (2007)

%*******************************************************DISK********************
\begin{longtable}[c]{cccccccccc}
\caption{NGC 338. Basic data}\label{GAL}  \\ 
\hline 
Type & D & $M_B$ & $M_R$ & $\mu_\mm{0,d}^\mm{c}$ & $h$ & $r_\mm{e,b}$
& $V_\mm{sys}$ & $PA$ & $i$ \\ 
& (Mpc) & (mag) & (mag) & (mag arcsec$^{-2}$) & (kpc) & (kpc)
& (km/s) & (deg) & (deg) \\
(1)&(2)&(3)&(4)&(5)&(6)&(7)&(8)&(9)&(10) \\ 
\hline
\endfirsthead 
\hline
\multicolumn{6}{c}{\small\slshape Basic data. 
Continued. } \\ \hline
Type & D & $M_B$ & $M_R$ & $\mu_\mm{0,d}^\mm{c}$ & $h$ & $r_\mm{e,b}$
& $V_\mm{sys}$ & $PA$ & $i$ \\ 
& (Mpc) & (mag) & (mag) & (mag arcsec$^{-2}$) & (kpc) & (kpc)
& (km/s) & (deg) & (deg) \\
(1)&(2)&(3)&(4)&(5)&(6)&(7)&(8)&(9)&(10) \\ 
\hline
\endhead 
\hline
Sab & 65.1 & -20.83 & -22.25 & 21.92 & 5.8 & 4.7 
& 4789 & 288 & 64 \tabularnewline
\hline
\end{longtable}

Columns: 
(1) Morphological type from NED). 
(2) Distance. 
(3), (4) absolute B-and R-band magnitudes 
(corrected for Galactic foreground extinction).
(5) R-band central disc surface brightness 
(corrected for Galactic foreground extinction and inclination effects).
(6) $R$-band disc scalelength.
(7) $R$-band bulge effective radius.
(8) Heliocentric systemic velocity.
(9) Position angle (north through east) of major axis. 
(10) Inclination angle.

\bigskip
\noindent
Фотометрия в $B$ и $R$ даёт очень маленькие значения поверхностной 
яркости диска в центре, а, соответственно, и поверхностной плотности. 
Использовать фотометрию в $I$.

\bigskip
\noindent
У галактики довольно много газа (большие значения поверхностной 
плотности). В диске, за пределами балджа, видны яркие узлы в области 
рукавов (SDSS). Наблюдается излучение ионизованного газа вплоть до 
45-47 arcsec.

\end{document}
