%\documentclass[a4paper,10pt]{article}
\documentclass[english,10pt]{article}

\usepackage[utf8]{inputenc}
\usepackage[russian,english]{babel}
%\usepackage{psfig,amsfonts}
\usepackage{array,longtable}
\usepackage{amssymb}
\usepackage{graphicx} 
\usepackage{graphics}
\setcounter{LTchunksize}{20}
\textheight 23.0cm
\textwidth 18.0cm

\hoffset=-2.5cm
%\voffset=-2.5cm
\baselineskip=22pt
\sloppy

\renewcommand{\refname}{ }
\renewcommand{\tablename}{Таблица}
\renewcommand{\figurename}{Рис.}

\def\bc{\begin{center}}
\def\ec{\end{center}}

\def\lb{\linebreak}
\def\be{\begin{equation}}
\def\ee{\end{equation}}
\def\beq{\begin{eqnarray}}
\def\eeq{\end{eqnarray}}
\def\bfig{\begin{figure}}
\def\efig{\end{figure}}
\def\bnum{\begin{enumerate}}
\def\enum{\end{enumerate}}
\def\ds{\displaystyle}
\def\mm{\mathrm}

\begin{document}

\noindent
Noordermeer E., van der Hulst J. M., 
``The stellar mass distribution in early-type disc galaxies: surface
photometry and bulge-disc decompositions'', 
MNRAS, 376, 1480-1512 (2007)

%*******************************************************DISK********************
\begin{longtable}[c]{cccccccccccccc}
\caption{NGC 5533. Structural parameters of the galaxy} \\ 
\hline 
band & scale & 
$\mu_\mm{e,b}$ & $\mu_\mm{e,b}^\mm{c}$ & $r_\mm{e,b}$ & 
$n$ & $m_\mm{b}$ & $M_\mm{b}$ & 
$\mu_\mm{0,d}$ & $\mu_\mm{0,d}^\mm{c}$ & $h$ & 
$m_\mm{d}$ & $M_\mm{d}$ & $B/D$ \\ 
& (kpc/$''$) & \multicolumn{2}{c}{(mag/$\Box^{''}$)} & ($''$) 
& & \multicolumn{2}{c}{(mag)} & 
\multicolumn{2}{c}{(mag/$\Box^{''}$)} & $''$ & 
\multicolumn{2}{c}{(mag)} & \\
(1)&(2)&(3)&(4)&(5)&(6)&(7)&(8)&(9)&(10)&(11)&(12)&(13)&(14) \\ 
\hline
\endfirsthead 
\hline
\multicolumn{14}{c}{\small\slshape Structural parameters of the galaxy. 
Continued. } \\ \hline
band & scale & 
$\mu_\mm{e,b}$ & $\mu_\mm{e,b}^\mm{c}$ & $r_\mm{e,b}$ & 
$n$ & $m_\mm{b}$ & $M_\mm{b}$ & 
$\mu_\mm{0,d}$ & $\mu_\mm{0,d}^\mm{c}$ & $h$ & 
$m_\mm{d}$ & $M_\mm{d}$ & $B/D$ \\ 
& (kpc/$''$) & \multicolumn{2}{c}{(mag/$\Box^{''}$)} & ($''$) 
& & \multicolumn{2}{c}{(mag)} & 
\multicolumn{2}{c}{(mag/$\Box^{''}$)} & & 
\multicolumn{2}{c}{(mag)} & \\
(1)&(2)&(3)&(4)&(5)&(6)&(7)&(8)&(9)&(10)&(11)&(12)&(13)&(14) \\
\hline
\endhead 
\hline
$R$ & 0.26 & 19.79 & 19.75 & 9.9 & 2.7 & 12.05 & -21.66 & 
20.78 & 21.27 & 34.4 & 11.69 & -22.02 & 0.72 \tabularnewline

$B$ & 0.26 & 21.52 & 21.45 & 9.9 & 2.7 & 13.78 & -19.96 & 
21.93 & 22.39 & 32.4 & 12.93 & -20.81 & 0.46 \tabularnewline
\hline
\end{longtable}

Columns: 
(1) Photometric band. 
(2) Conversion factor to convert arcsecs into kpc.
(3) Bulge effective surface brightness. 
(4) Idem, but corrected for galactic foreground extinction.
(5) Effective radius of the bulge, given in arcsec.
(6) S\`ersic index. 
(7) Bulge total apparent magnitude.
(8) Bulge total absolute magnitude.
(9) Disc central surface brightness. 
(10) Idem, but corrected for galactic foreground extinction.
(11) Disc scalelength, given in arcsec.
(12) Disc total apparent magnitude.
(13) Disc total absolute magnitude. 
(14) The ratio of the bulge to disc luminosities. 

\bigskip
\noindent
Noordermeer E., van der Hulst J.M., Sancisi R., 
Swaters R. S., and van Albada T.S., 
``The mass distribution in early-type disc galaxies: declining rotation
curves and correlations with optical properties'', 
MNRAS, 376, 1513-1546 (2007)

%*******************************************************DISK********************
\begin{longtable}[c]{cccccccccc}
\caption{NGC 5533. Basic data} \\ 
\hline 
Type & D & $M_B$ & $M_R$ & $\mu_\mm{0,d}^\mm{c}$ & $h$ & $r_\mm{e,b}$
& $V_\mm{sys}$ & $PA$ & $i$ \\ 
& (Mpc) & (mag) & (mag) & (mag/$\Box^{''}$) & (kpc) & (kpc)
& (km/s) & (deg) & (deg) \\
(1)&(2)&(3)&(4)&(5)&(6)&(7)&(8)&(9)&(10) \\ 
\hline
\endfirsthead 
\hline
\multicolumn{10}{c}{\small\slshape Basic data. 
Continued. } \\ \hline
Type & D & $M_B$ & $M_R$ & $\mu_\mm{0,d}^\mm{c}$ & $h$ & $r_\mm{e,b}$
& $V_\mm{sys}$ & $PA$ & $i$ \\ 
& (Mpc) & (mag) & (mag) & (mag/$\Box^{''}$) & (kpc) & (kpc)
& (km/s) & (deg) & (deg) \\
(1)&(2)&(3)&(4)&(5)&(6)&(7)&(8)&(9)&(10) \\ 
\hline
\endhead 
\hline
SA(rs)ab & 54.3 & -21.22 & -22.62 & 21.27 & 9.1 & 2.6 
& 3858 & 24-45 & 53 \tabularnewline
\hline
\end{longtable}

Columns: 
(1) Morphological type from NED). 
(2) Distance. 
(3), (4) absolute B-and R-band magnitudes 
(corrected for Galactic foreground extinction).
(5) R-band central disc surface brightness 
(corrected for Galactic foreground extinction and inclination effects).
(6) $R$-band disc scalelength.
(7) $R$-band bulge effective radius.
(8) Heliocentric systemic velocity.
(9) Position angle (north through east) of major axis. 
(10)Inclination angle.

\bigskip
\noindent
M\'{e}ndez-Abreu J., Aguerri J. A. L., Corsini E. M., 
and Simonneau E., 
``Structural properties of disk galaxies. I. The intrinsic 
equatorial ellipticity of bulges'', 
A\&A, 478, 353-369 (2008)

%*******************************************************DISK********************
\begin{longtable}[c]{cccccccccccc}
\caption{NGC 5533. Structural parameters of the galaxy} \\ 
\hline 
band & D & $V_{3K}$ & 
$\mu_\mm{e,b}$ & $r_\mm{e,b}$ & $n$ & $q_\mm{b}$ & PA$_\mm{b}$ & 
$\mu_\mm{0,d}$ & $h$ & $q_\mm{d}$ & PA$_\mm{d}$ \\ 
& (Mpc) & (km/s) & 
(mag/$\Box^{''}$) & ($''$) & & & (deg) & 
(mag/$\Box^{''}$) & ($''$) & & (deg) \\
(1)&(2)&(3)&(4)&(5)&(6)&(7)&(8)&(9)&(10)&(11)&(12) \\ 
\hline
\endfirsthead 
\hline
\multicolumn{12}{c}{\small\slshape Structural parameters of the galaxy. 
Continued. } \\ \hline
band & D & $V_{3K}$ & 
$\mu_\mm{e,b}$ & $r_\mm{e,b}$ & $n$ & $q_\mm{b}$ & PA$_\mm{b}$ & 
$\mu_\mm{0,d}$ & $h$ & $q_\mm{d}$ & PA$_\mm{d}$ \\ 
& (Mpc) & (km/s) & 
(mag/$\Box^{''}$) & ($''$) & & & (deg) & 
(mag/$\Box^{''}$) & ($''$) & & (deg) \\
(1)&(2)&(3)&(4)&(5)&(6)&(7)&(8)&(9)&(10)&(11)&(12) \\
\hline
\endhead 
\hline
$J$ & 54.0 & 4051 &
17.70 & 5.1 & 2.34 & 0.66 & 23.6 & 
18.03 & 14.1 & 0.64 & 26.3 \tabularnewline
\hline
\end{longtable}

Columns:
(1) Photometric band.
(2) Distance, obtained as $V_{3K}/H_0$ with $H_0=$75 km\,s$^{-1}$.
(3) Radial velocity with respect to the CMB from LEDA.
(4) Bulge effective surface brightness. 
(5) Effective radius of the bulge, given in arcsec.
(6) S\`ersic index. 
(7) Axis ratio of the bulge.
(8) Position angle of the bulgescale lenth of the disc.
(9) Disc central surface brightness. 
(10) Disc scalelength, given in arcsec.
(11) Axis ratio of the disc.
(12) Position angle of the disc. 

\bigskip
\noindent
Фотометрия в $B$ и $R$ даёт согласованные значения центральной 
поверхностной плотности для диска, но они не очень большие 
при отношении $(M/L)_R \approx 2.11$ (по цвету $B-R$) 
$\Sigma_{0,\mathrm{d}} = 138.4 \, \, M_\odot/$пк$^2$. 
Есть динамическая оценка $(M/L)_R$ (из кривой вращения, 
Noordermeer, thesis) --- от 0 до 5 
(5 --- для ``максимального'' диска), если используется 
NFW модель тёмного гало. Для максимального диска 
центральная поверхностная плотность --- 328.0 $M_\odot/$пк$^2$. 

\bigskip
\noindent
Фотометрия в полосе $J$ (M\'{e}ndez-Abreu et al., 2008) 
не согласуется с фотометрией в оптике, т.е. приводит к 
значениям поверхностной плотности диска, отличающимся от тех, что 
получаются для полосы $R$. Центральная поверхностная яркость --- 
746.3 $L_\odot/$пк$^2$, центральная поверхностная плотность 
того же порядка при $(M/L)_J \approx 1.0$.

\bigskip
\noindent
Декомпозиция галактики противоречивая. В полосе $J$ масштаб 
диска явно занижен, а поверхностная яркость завышена.

\bigskip
\noindent
Галактика типа Sab. 
В области вне яркого балджа видны туго закрученные спирали, в них 
голубые уярчения (SDSS). 
Наблюдается излучение в линиях ионизованного водорода 
вплоть до 50 arcsec.

\bigskip
\noindent
У галактики довольно много газа. 
Интегрально $M_\mm{HI} = 2.63 \cdot 10^{10} M_\odot$. 
Значения поверхностной плотности в пределах оптического 
радиуса галактики ($\sim 180$ arcsec) умеренно большие.
Профиль распределения плотности водорода монотонно убывающий.

\bigskip
\noindent
Кривая вращения по газу быстро выходит на плато.


\end{document}
